\documentclass[11pt, oneside, article]{memoir}

\settrims{0pt}{0pt} % page and stock same size
\settypeblocksize{*}{35pc}{*} % {height}{width}{ratio}
\setlrmargins{*}{*}{1} % {spine}{edge}{ratio}
\setulmarginsandblock{1.1in}{1.4in}{*} % height of typeblock computed
\setheadfoot{\onelineskip}{2\onelineskip} % {headheight}{footskip}
\setheaderspaces{*}{1.5\onelineskip}{*} % {headdrop}{headsep}{ratio}
\checkandfixthelayout

\usepackage[centertags,sumlimits,intlimits,namelimits,reqno]{amsmath}
\usepackage{latexsym}
\usepackage{footmisc}
\usepackage{amsfonts,amsthm,amssymb}
\usepackage{mathtools}
\usepackage[cal=euler,scr=rsfso]{mathalfa}
\usepackage[boxslash]{stmaryrd}
\usepackage{newpxtext}
%\usepackage{exscale}
\usepackage{enumitem}
\usepackage{ifthen}
%\usepackage[T1]{fontenc}
%\usepackage{breakcites}
\usepackage[colorlinks,linkcolor=darkblue,citecolor=darkblue,urlcolor=darkblue,breaklinks=true]{hyperref}
\usepackage{tikz}
\usepackage[capitalize]{cleveref}
\usepackage[backend=biber, bibencoding=utf8, maxbibnames = 10, style = alphabetic]{biblatex}
\DeclareMathAlphabet{\mathpzc}{OT1}{pzc}{m}{it}
\usepackage{xstring}
\usepackage{ebproof}
\usepackage[export]{adjustbox} %vertical align includegraphics
%\usepackage{scalefnt}
%\usepackage{anyfontsize} %arbitrary font size
\usepackage{graphicx}

\usepackage[color=white, textsize=footnotesize]{todonotes}

\presetkeys%
    {todonotes}%
    {inline}{}
    
%% code from mathabx.sty and mathabx.dcl
\DeclareFontFamily{U}{mathx}{\hyphenchar\font45}
\DeclareFontShape{U}{mathx}{m}{n}{
      <5> <6> <7> <8> <9> <10>
      <10.95> <12> <14.4> <17.28> <20.74> <24.88>
      mathx10
      }{}
\DeclareSymbolFont{mathx}{U}{mathx}{m}{n}
\DeclareFontSubstitution{U}{mathx}{m}{n}
\DeclareMathAccent{\widecheck}{0}{mathx}{"71}

% amsmath %
	\allowdisplaybreaks
	
% cleveref %
  \newcommand{\creflastconjunction}{, and\nobreakspace} % serial comma
  \definecolor{darkblue}{rgb}{0,0,0.7} 


% tikz %
  \usetikzlibrary{ 
  	cd,
  	math,
  	decorations.markings,
		decorations.pathreplacing,
  	positioning,
	 	circuits.logic.US,
 		arrows.meta,
  	shapes,
		shadows,
		shadings,
  	calc,
  	fit,
  	quotes,
  	intersections,
  }

  \input{tikz-stuff}
	
% biblatex %
  \addbibresource{Library20190610.bib} 

% enumitem %
  \setlist{noitemsep, nolistsep}
	\setlist[description]{leftmargin=0em, itemindent=2em}
		
%-------------------------------------------------------------------------
\theoremstyle{plain}
\newtheorem{theorem}{Theorem}[chapter] %change [] to chapter if we want to change global numbering
\newtheorem{proposition}[theorem]{Proposition}
\newtheorem{corollary}[theorem]{Corollary}
\newtheorem{lemma}[theorem]{Lemma}
\newtheorem{conjecture}[theorem]{Conjecture}

\theoremstyle{definition}
\newtheorem{definition}[theorem]{Definition}
\newtheorem{construction}[theorem]{Construction}
\newtheorem{notation}[theorem]{Notation}
\newtheorem{axiom}{Axiom}
\newtheorem*{axiom*}{Axiom}

\theoremstyle{remark}
\newtheorem{example}[theorem]{Example}
\newtheorem{remark}[theorem]{Remark}
\newtheorem{warning}[theorem]{Warning}
\newtheorem{question}[theorem]{Question}

\crefalias{chapter}{section}

%------------------Begin author macros-----------------------

\newcommand{\Set}[1]{\mathrm{#1}}%a named set
\newcommand{\ord}[1]{\underline{#1}}%a natural number, considered as a finite set
\newcommand{\const}[1]{\mathtt{#1}}%a constant, named element of a set, sort of thing
\newcommand{\cat}[1]{\mathcal{#1}}%a generic category
\newcommand{\ccat}[1]{\mathbb{#1}}%a generic category
\newcommand{\Cat}[1]{{\mathsf{#1}}}%a named category
\newcommand{\CCat}[1]{\mathbb{\StrLeft{#1}{1}}\Cat{\StrGobbleLeft{#1}{1}}}%a named category; does not seem to work in section headers...
\newcommand{\funn}[1]{\mathrm{#1}}%a function
\newcommand{\funr}[1]{\mathcal{#1}}%a generic functor
\newcommand{\Funr}[1]{\mathsf{#1}}%a named functor
\newcommand{\ffunr}[1]{\mathbf{#1}}%a generic 2-functor


\DeclareMathOperator{\ob}{\Set{Ob}}
\DeclareMathOperator{\id}{id}
\DeclareMathOperator{\dom}{dom}
\DeclareMathOperator{\cod}{cod}
\DeclareMathOperator{\Hom}{Hom}
\DeclareMathOperator*{\colim}{colim}
\DeclareMathOperator{\coker}{coker}
\DeclareMathOperator{\im}{im}
\DeclareMathOperator{\inc}{inc}
\DeclarePairedDelimiter{\pair}{\langle}{\rangle}
\DeclarePairedDelimiter{\unary}{{\langle\,}}{{\,\rangle}}
\DeclarePairedDelimiter{\copair}{[}{]}
\DeclarePairedDelimiter{\classify}{{\raisebox{1pt}{$\ulcorner$}}}{{\raisebox{1pt}{$\urcorner$}}}
\DeclarePairedDelimiter{\church}{\xxbracket}{\rrbracket}

\newcommand{\tn}[1]{\textnormal{#1}}
\newcommand{\op}{^{\tn{op}}}
\newcommand{\inv}{^{-1}}
\newcommand{\tpow}[1]{^{\otimes #1}}

\newcommand{\finset}{\Cat{FinSet}}
\newcommand{\smset}{\Cat{Set}}
\newcommand{\poset}{\Cat{Poset}}
\newcommand{\smf}{\Cat{SMF}}
\newcommand{\ssmc}{\CCat{SMC}}

\renewcommand{\aa}{\mathbb{A}} %Not sure what old \aa did
\renewcommand{\ll}{\mathbb{L}} %Old \ll is <<
\newcommand{\nn}{\mathbb{N}}
\newcommand{\pp}{\mathbb{P}}
\newcommand{\qq}{\mathbb{Q}}

\newcommand{\mob}[1]{#1_0}

\newcommand{\rel}{\Cat{Rel}}
\newcommand{\cospan}{\Cat{Cospan}}

\newcommand{\ust}[1]{#1^{*}}
\newcommand{\lsh}[1]{#1_{!}}


\newcommand{\cp}{\mathbin{\fatsemi}}
\newcommand{\cocolon}{:\!}
\newcommand{\iso}{\cong}
\newcommand{\too}{\longrightarrow}
\newcommand{\tto}{\rightrightarrows}
\newcommand{\To}[1]{\xrightarrow{#1}}
\newcommand{\Too}[1]{\To{\;\;#1\;\;}}
\newcommand{\from}{\leftarrow}
\newcommand{\From}[1]{\xleftarrow{#1}}
\newcommand{\Fromm}[1]{\xleftarrow{\;\;#1\;\;}}
\newcommand{\tofrom}{\leftrightarrows}
\newcommand{\surj}{\twoheadrightarrow}
\newcommand{\inj}{\rightarrowtail}
\newcommand{\Surj}[1]{\overset{#1}{\twoheadrightarrow}}
\newcommand{\Inj}[1]{\overset{#1}{\rightarrowtail}}
\newcommand{\frsurj}{\twoheadleftarrow}
\newcommand{\frinj}{\leftarrowtail}
\newcommand{\ul}[1]{\underline{#1}}
\renewcommand{\ss}{\subseteq}


\newcommand{\qqand}{\qquad\text{and}\qquad}
\newcommand{\qand}{\quad\text{and}\quad}

\newcommand{\hide}[2][]{#1}

\newcommand{\adjphantom}[3][-.6pt]{\ar[#2, phantom, "#3" yshift=#1]}
\newcommand{\adj}[5][30pt]{%[size] Cat L, Left, Right, Cat R.
\begin{tikzcd}[ampersand replacement=\&, column sep=#1]
  #2\ar[r, shift left=5pt, "{#3}"]\adjphantom{r}{\Rightarrow}\&
  #5\ar[l, shift left=5pt, "{#4}"]
\end{tikzcd}
}

\newcommand{\adjr}[5][30pt]{%[size] Cat R, Right, Left, Cat L.
\begin{tikzcd}[ampersand replacement=\&, column sep=#1]
  #2\ar[r, shift left=5pt, "{#3}"]\adjphantom{r}{\Leftarrow}\&
  #5\ar[l, shift left=5pt, "{#4}"]
\end{tikzcd}
}

\newcommand{\pb}[1][very near start]{\ar[dr, phantom, #1, "\lrcorner"]}
\newcommand{\po}[1][very near start]{\ar[ul, phantom, #1, "\ulcorner"]}

\setlength\tabcolsep{3pt}
\linespread{1.10}

%================ Document ================%
\begin{document}   

\title{Supplying bells and whistles in monoidal categories}
\author{Brendan Fong$^*$ \and David I.\ Spivak\thanks{Spivak and Fong acknowledge support from AFOSR grants FA9550-17-1-0058 and FA9550-19-1-0113.}}
  
\maketitle

\begin{abstract}

\end{abstract}

%======== Chapter ========%
\chapter{Introduction}

Many symmetric monoidal categories $\cat{C}$ have the property that each object $c\in\cat{C}$ is equipped with a certain algebraic structure---say that of a monoid or a comonoid---in a way that is compatible with the monoidal structure.

For example, consider the category $\rel$ of relations between sets. It has a symmetric monoidal structure $(1, \otimes, \sigma)$ coming from the Cartesian monoidal structure of $\smset$. This is not a Cartesian monoidal structure on $\rel$, e.g.\ 1 is not terminal. And yet each object $r\in\rel$ is equipped morphisms $\delta_r\colon r\to r\otimes r$ and $\epsilon_r\colon r\to 1$, which satisfy the same properties that a diagonal and a terminal morphism do. Namely, the diagrams expressing commutativity, unitality, and associativity commute:
\[
\begin{tikzcd}[column sep=small]
	&r\ar[dl, "\delta"']\ar[dr, "\delta"]\\
	r\otimes r\ar[rr, "\sigma"']&&
	r\otimes r
\end{tikzcd}
\qquad
\begin{tikzcd}[column sep=small]
	r\ar[rr, equal]\ar[dr, "\delta"']&&
	r\\&
	r\otimes r\ar[ur, "\epsilon\otimes r"']
\end{tikzcd}
\qquad
\begin{tikzcd}
	r\ar[r, "\delta"]\ar[d, "\delta"']&
	r\otimes r\ar[d, "r\otimes\delta"]\\
	r\otimes r\ar[r, "\delta\otimes r"']&
	r\otimes r\otimes r
\end{tikzcd}
\]
In string diagrams, the maps $\delta$ and $\epsilon$ can be drawn as:
\[
\begin{tikzpicture}[WD]
	\node[bb={1}{0}] (eta') {$\epsilon$};
	\draw (eta'_in1) to +(-.8,0);
	\node[bb={1}{2}, right=4 of eta'] (mu') {$\delta$};
	\draw (mu'_in1) -- +(-.8,0);
	\draw (mu'_out1) -- +(.8,0);
	\draw (mu'_out2) -- +(.8,0);
\end{tikzpicture}
\]
and the equations can be drawn as:
\[
\begin{tikzpicture}
	\node (Q11) {
	\begin{tikzpicture}[WD]
		\node[bb={1}{2}] (a) {};
		\coordinate (a1) at ($(a_out1)+(1,0)$);
		\coordinate (a2) at ($(a_out2)+(1,0)$);
		\draw (a_out2) to[out=0, in=180] (a1);
		\draw (a_out1) to[out=0, in=180] (a2);
		\draw (a_in1) -- +(-.5,0);
	\end{tikzpicture}
	};
	\node (Q12) [right=.8 of Q11] {
	\begin{tikzpicture}[WD]
		\node[bb={1}{2}] (a) {};
		\draw (a_out1) -- +(.5,0);
		\draw (a_out2) -- +(.5,0);
		\draw (a_in1) -- +(-.5,0);
	\end{tikzpicture}
	};
	\node[label=above:{\tiny commutative}] at ($(Q11.east)!.5!(Q12.west)$) {$=$};
%
	\node (Q21) [right=1 of Q12] {
  \begin{tikzpicture}[WD]
  	\node[bb={1}{2}] (a1) {};
  	\node[bb={1}{0}, right=.5 of a1_out1] (a2) {};
  	\draw (a2_in1) -- (a1_out1);
  	\draw (a1_out2) -- +(2,0);
  	\draw (a1_in1) -- +(-.5,0);
	\end{tikzpicture}
	};
	\node (Q22) [right=.8 of Q21] {
	\begin{tikzpicture}[WD]
		\draw (0,0) -- (2,0);
	\end{tikzpicture}
	};	
	\node[label=above:{\tiny unital}] at ($(Q21.east)!.5!(Q22.west)$) {$=$};
%
	\node (Q31) [right=1 of Q22] {
	\begin{tikzpicture}[WD]
		\node[bb={1}{2}] (a1) {};
		\node[bb={1}{2}, minimum height=1ex, right=.5 of a1_out1] (a2) {};
		\draw (a2_in1) -- (a1_out1);
		\draw (a1_out2) -- +(2,0);
		\draw (a2_out1) -- +(.5,0);
		\draw (a2_out2) -- +(.5,0);
		\draw (a1_in1) -- +(-.5,0);
	\end{tikzpicture}
	};
	\node (Q32) [right=.8 of Q31] {
	\begin{tikzpicture}[WD]
		\node[bb={1}{2}] (a1) {};
		\node[bb={1}{2}, minimum height=1ex, right=.5 of a1_out2] (a2) {};
		\draw (a2_in1) -- (a1_out2);
		\draw (a1_out1) -- +(2,0);
		\draw (a2_out1) -- +(.5,0);
		\draw (a2_out2) -- +(.5,0);
		\draw (a1_in1) -- +(-.5,0);
	\end{tikzpicture}
	};
	\node[label=above:{\tiny associative}] at ($(Q31.east)!.5!(Q32.west)$) {$=$};
\end{tikzpicture}
\]
Not only is every object in $\rel$ equipped with these structures, but they are compatible under the monoidal structure. First, the algebraic structure assigned to the monoidal unit is trivial: $\epsilon_I=\id_I$ and $\delta_I=\rho_I=\lambda_I$, where $\rho$ and $\lambda$ are the right and left unitors. Second, the algebraic structures interact nicely with the monoidal product:
\[
  \epsilon_{r\otimes s}=\epsilon_r\otimes\epsilon_s
  \qqand
  \delta_{r\otimes s}=(\delta_r\otimes\delta_s)\cp(r\otimes\sigma\otimes s).
\]
One notices immediately the need for a symmetry isomorphism $\sigma$. In pictures:
\[
\begin{tikzpicture}
	\node (P1) {
	\begin{tikzpicture}[WD]
		\node[bb={2}{0}] (rs) {$\epsilon_{r\otimes s}$};
		\draw (rs_in1) to node[above, font=\tiny] {$r$} +(-1, 0);
		\draw (rs_in2) to node[below, font=\tiny] {$s$} +(-1, 0);		
	\end{tikzpicture}
	};
	\node (P2) [right=.7 of P1] {
	\begin{tikzpicture}[WD]
		\node[bb={1}{0}] (r) {$\epsilon_r$};
		\node[bb={1}{0}, below=.5 of r] (s) {$\epsilon_s$};
		\draw (r_in1) to node[above, font=\tiny] {$r$} +(-1, 0);
		\draw (s_in1) to node[below, font=\tiny] {$s$} +(-1, 0);
	\end{tikzpicture}
	};
	\node at ($(P1.east)!.5!(P2.west)$) {$=$};
%
	\node (P3) [right=2 of P2] {
	\begin{tikzpicture}[WD, bb port sep=.7]
		\node[bb={2}{4}] (rs) {$\delta_{r\otimes s}$};
		\begin{scope}[font=\tiny]
  		\draw (rs_in1) to node[above] {$r$} +(-1, 0);
  		\draw (rs_in2) to node[below] {$s$} +(-1, 0);
  		\draw (rs_out1) to node[above=-1pt] {$r$} +(1, 0);
  		\draw (rs_out2) to node[above=-1pt] {$s$} +(1, 0);
  		\draw (rs_out3) to node[below=-2pt] {$r$} +(1, 0);
  		\draw (rs_out4) to node[below=-2pt] {$s$} +(1, 0);
		\end{scope}
	\end{tikzpicture}
	};
	\node (P4) [right=.7 of P3] {
	\begin{tikzpicture}[WD, bb port sep=.7]
		\node[bb={1}{2}] (r) {$\delta_r$};
		\node[bb={1}{2}, below=.5 of r] (s) {$\delta_s$};
		\coordinate (r2) at ($(s_out1)+(2,0)$);
		\coordinate (s1) at ($(r_out2)+(2,0)$);
		\begin{scope}[font=\tiny]
			\draw (r_in1) to node[above] {$r$} +(-1, 0);
			\draw (s_in1) to node[below] {$s$} +(-1, 0);
			\draw (r_out1) -- +(2.5, 0) to node[above=-1pt] {$r$} +(.5, 0);
			\draw (r_out2) -- +(1, 0) to[out=0, in=180] (r2) to node[below=-1pt] {$r$} +(.5, 0);
			\draw (s_out1) -- +(1, 0) to[out=0, in=180] (s1) to node[above=-2pt] {$s$} +(.5, 0);
			\draw (s_out2) -- +(2.5, 0) to node[below=-2pt] {$s$} +(.5, 0);
    \end{scope}		
	\end{tikzpicture}
	};
	\node at ($(P3.east)!.5!(P4.west)$) {$=$};
\end{tikzpicture}
\]
In the language of this paper, we would say that $\rel$ \emph{supplies comonoids}: each object in $\rel$ is equipped with a comonoid structure in a way compatible with $1$ and $\otimes$.

In general, the compatible algebraic structures assigned to each object in $\cat{C}$ can be understood as coming from a prop $\pp$. A prop is a strict monoidal category whose monoid of objects is $(\nn,0,+)$. In the case of comonoids, the relevant prop is the skeleton of $\finset\op$. Indeed, $\epsilon$ and $\delta$ represent the (opposites of) the unique functions $\varnothing\to\{1\}$ and $\{1,2\}\to\{1\}$, respectively.

The first goal of this note is to give a general definition of supply. If $\pp$ is a prop and $\cat{C}$ is a symmetric monoidal category, we say carefully what it means for $\cat{C}$ to supply the algebraic structure encoded in $\pp$. 

We give a number of simple consequences. For example, a supply of $\pp$ and a prop functor $\pp'\to\pp$ gives rise to a supply of $\pp'$. We also discuss what it means for a strong monoidal functor $\cat{C}\to\cat{D}$ to \emph{preserve supplies}, i.e.\ to send a given supply of $\pp$ in $\cat{C}$ to a given supply of $\pp$ in $\cat{D}$. We show that if $\cat{C}$ and $\cat{D}$ both supply $\pp$ then so does their biproduct $\cat{C}\oplus\cat{D}$, and the projections and coprojections preserve supplies. It does not appear to be sufficiently well-known that the 2-category of symmetric monoidal categories has (strong) biproducts, so we prove that as well.

We also discuss what it means for various maps in $(\cat{C},I,\otimes)$ to be \emph{homomorphisms} for the supplied structure. For example, if $\cat{C}$ is Cartesian monoidal then it supplies comonoids and every morphism $f\colon c\to d$ is a comonoid homomorphism in the sense that the following diagrams commute:
\[
\begin{tikzcd}[column sep=small]
	c\ar[rr, "f"]\ar[rd, "\epsilon_c"']&&
	d\ar[dl, "\epsilon_d"]\\&
	I
\end{tikzcd}
\hspace{.7in}
\begin{tikzcd}
	c\ar[r, "f"]\ar[d, "\delta_c"']&
	d\ar[d, "\delta_d"]\\
	c\otimes c\ar[r, "f\otimes f"']&
	d\otimes d
\end{tikzcd}
\]
Again in pictures:
\[
\begin{tikzpicture}
	\node (P1) {
	\begin{tikzpicture}[WD]
		\node[bb={1}{1}] (f) {$f$};
		\node[bb={1}{0}, right=.5 of f] (e) {$\epsilon_d$};
		\draw (f_in1) -- +(-1, 0);
		\draw (f_out1) -- (e_in1);
	\end{tikzpicture}
	};
	\node (P2) [right=.7 of P1] {
	\begin{tikzpicture}[WD]
		\node[bb={1}{0}] (e) {$\epsilon_c$};
		\draw (e_in1) -- +(-1, 0);
	\end{tikzpicture}
	};
	\node at ($(P1.east)!.5!(P2.west)$) {$=$};
%
	\node (P3) [right=2 of P2] {
	\begin{tikzpicture}[WD]
		\node[bb={1}{1}] (f) {$f$};
		\node[bb={1}{2}, right=.5 of f] (d) {$\delta_d$};
		\draw (f_in1) -- +(-1, 0);
		\draw (f_out1) -- (d_in1);
		\draw (d_out1) -- +(1, 0);
		\draw (d_out2) -- +(1, 0);
  \end{tikzpicture}	
	};
	\node (P4) [right=.7 of P3] {
	\begin{tikzpicture}[WD]
		\node[bb port sep=1.2, bb={1}{2}] (d) {$\delta_c$};
		\node[bb={1}{1}, right=.5 of d_out1, font=\tiny] (f1) {$f$};
		\node[bb={1}{1}, right=.5 of d_out2, font=\tiny] (f2) {$f$};
		\draw (d_in1) -- +(-1, 0);
		\draw (d_out1) -- (f1_in1);
		\draw (d_out2) -- (f2_in1);
		\draw (f1_out1) -- +(1, 0);
		\draw (f2_out1) -- +(1, 0);
	\end{tikzpicture}
	};
	\node at ($(P3.east)!.5!(P4.west)$) {$=$};
\end{tikzpicture}
\]
Again, these equations hold in any Cartesian monoidal category, but they \emph{do not hold} in $\rel$. (As an example, take $c=d=1$, take $f$ to be the empty relation $\varnothing\ss 1\times 1$, and note that $(f\cp\epsilon_d)\colon 1\to 1$ is empty whereas $\epsilon_c$ is not.)

We will show that the morphisms in $\cat{C}$ that are homomorphisms for the $\pp$-structure always form a monoidal subcategory. The second goal of the paper is to prove that the coherence isomorphisms---the associator and unitors---are always homomorphisms for the $\pp$-structure.



%======== Chapter ========%
\chapter{Notation and background}

%==== Section ====%
\section{Basic notation}
For a natural number $n\in\nn$ we denote the corresponding ordinal by $\ord{n}=\{1,\ldots,n\}\in\smset$. Given a finite set $S$, we write $|S|\in\nn$ for its cardinality.

We denote composition of $f\colon a\to b$ and $g\colon b\to c$ by $(f\cp g)\colon a\to c$, i.e.\ we use diagrammatic order. When using application order, we write $g\circ f$. When $c$ is an object we denote the identity morphism on it either by $c$ or by $\id_c$.

Given morphisms $f\colon A\to C$ and $g\colon B\to C$ in a category with coproducts, we denote the copairing of $f$ and $g$ by $\copair{f,g}\colon(A\sqcup B)\to C$. Similarly if the category has products, we denote the pairing of $f\colon A\to B$ and $g\colon A\to C$ by $\pair{f,g}\colon A\to B\times C$.

%==== Section ====%
\section{Symmetric monoidal categories and coherence}

Suppose $(\cat{C}, I, \otimes)$ is a symmetric monoidal category, $m\in\nn$ is a natural number, and $c\colon\ord{m}\to\cat{C}$ is a family of objects in $\cat{C}$. We denote
\[
  \bigotimes c\coloneqq\bigotimes_{i\in\ord{m}}c(i)\coloneqq
  \big((c(1)\otimes c(2))\cdots\big)\otimes c(m)
\]
with the convention that when $m=0$ and $!\colon\ord{0}\to \cat{C}$ the unique function, we put $\bigotimes != I$. If $c(i)=c(j)$ for all $i,j\in\ord{m}$, we denote this by $c\tpow{m}\coloneqq\bigotimes_{i\in\ord{m}}c$. We take this to be the canonical parenthesation, so $c\otimes d\otimes e$ denotes $(c\otimes d)\otimes e$. 

Many of our results will rely on Mac Lane's coherence theorem for symmetric monoidal categories \cite[Theorem XI.1]{maclane:1998a}, which says the following. For any two ways to arrange parentheses and monoidal units into a word with $n$ placeholders for objects in $\cat{C}$, and for each permutation of $n$ letters, there is an associated natural isomorphism, which Mac Lane calls the \emph{canonical isomorphism}, between the resulting functors $\cat{C}^n\to\cat{C}$, and composites and tensor products of canonical isomorphisms are again canonical. For example, everything we called a symmetry isomorphisms $\sigma$ in \cref{eqn.symmetry} is one of these canonical isomorphisms. Finally, nothing changes if we replace $\cat{C}$ by a symmetric monoidal category $\cat{C}$, because by definition all diagrams of 2-cells commute in $\cat{C}$.


If $m,n\in\nn$ are natural numbers, and $c\colon \ord{m}\times \ord{n}\to\cat{C}$ is a family of objects in $\cat{C}$, we also have a natural isomorphism
\begin{equation}\label{eqn.symmetry}
\sigma\colon
\bigotimes_{i\in\ord{m}}\bigotimes_{j\in\ord{n}}c(i,j)\Too{\cong}
\bigotimes_{j\in\ord{n}}\bigotimes_{i\in\ord{m}}c(i,j).
\end{equation}
We refer to $\sigma$ as the \emph{symmetry} isomorphism, though note that it involves associators and unitors too, not just the symmetric braiding. We will be interested in two particular cases of the symmetry isomorphism \cref{eqn.symmetry}, namely for $m=2$ and $m=0$ and any $n\in\nn$:
\[\sigma\colon c_1\tpow{n}\otimes c_2\tpow{n}\Too{\cong}(c_1\otimes c_2)\tpow{n}
\qqand
\sigma\colon I\Too{\cong} I\tpow{n}
\]

%==== Section ====%
\section{The 2-category $\mathbb{S}\Cat{MC}$}

\begin{theorem}
The 2-category $\ssmc$ of symmetric monoidal categories, strong monoidal functors, and monoidal natural transformations has strong 2-biproducts.
\end{theorem}
\begin{proof}
The terminal category is symmetric monoidal, and it is a zero object
Let $\cat{C}$ and $\cat{D}$ be symmetric monoidal categories. We will show that their product $\cat{C}\times\cat{D}$ as categories is in fact a symmetric monoidal category, and that it is their biproduct; we denote it $\cat{C}\oplus\cat{D}\coloneqq\cat{C}\times\cat{D}$.

As a symmetric monoidal structure on $\cat{C}\oplus\cat{D}$, take $(I,I)$ to be the monoidal unit and $(c_1,d_1)\otimes(c_2,d_2)\coloneqq(c_1\otimes c_2,d_1\otimes d_2)$ to be the monoidal product. The associators, unitors, and braiding are given pointwise.

The functor $\pair{\cat{C},I}\colon\cat{C}\to\cat{C}\oplus\cat{D}$ sending $c\mapsto (c,I)$ is clearly strong monoidal. We claim that it and $\pair{I,\cat{D}}$ form the coprojections under which $\cat{C}\oplus\cat{D}$ is the coproduct. Indeed, given strong monoidal functors $F\colon\cat{C}\to\cat{X}$ and $G\colon\cat{D}\to\cat{X}$, define $\copair{F,G}\colon\cat{C}\oplus\cat{D}\to\cat{X}$ by $\copair{F,G}(c,d)\coloneqq F(c)\otimes G(d)$, and similarly for morphisms. This is strong monoidal using the isomorphisms
\begin{align*}
	\copair{F,G}(c_1,d_1)\otimes\copair{F,G}(c_2,d_2)&=
	F(c_1)\otimes G(d_1)\otimes F(c_2)\otimes G(d_2)\\&\cong
	F(c_1)\otimes F(c_2)\otimes G(d_1)\otimes G(d_2)\\&\cong
	\copair{F,G}\big((c_1,d_1)\otimes(c_2,d_2)\big).
\end{align*}
where the first isomorphism is the braiding in $\cat{C}$ and the second isomorphism uses the (strong) laxators from $F$ and $G$. It is straightforward to check that this satisfies the necessary axioms to be a laxator%
\hide[.]{
, e.g.\ the commutativity of the following diagram
\[
\begin{tikzcd}
	Fc_1\otimes Gd_1\otimes Fc_2\otimes Gd_2\otimes Fc_3\otimes Gd_3\ar[r, "\sigma"]\ar[d, "\sigma"']&
	Fc_1\otimes Gd_1\otimes Fc_2\otimes Fc_3\otimes Gd_2\otimes Gd_3\ar[d]\\
	Fc_1\otimes Fc_2\otimes Gc_1\otimes Gc_2\otimes Fc_3\otimes Gd_3\ar[d]&
	Fc_1\otimes Gd_1\otimes F(c_2\otimes c_3)\otimes G(d_2\otimes d_3)\ar[d, "\sigma"]\\
	F(c_1\otimes c_2)\otimes G(d_1\otimes d_2)\otimes Fc_3\otimes Gd_3\ar[d, "\sigma"']&
	Fc_1\otimes F(c_2\otimes c_3)\otimes Gd_1\otimes G(d_2\otimes d_3)\ar[d]\\
	F(c_1\otimes c_2)\otimes Fc_3\otimes G(d_1\otimes d_2)\otimes Gd_3\ar[r]&
	F(c_1\otimes c_2\otimes c_3)\otimes G(d_1\otimes d_2\otimes d_3)	
\end{tikzcd}
\]
}
It is also easy to check that the unitors provide natural isomorphisms
\begin{equation}\label{eqn.coproduct_smc}
\begin{tikzcd}[sep=large]
	\cat{C}\ar[r, "\pair{\cat{C},I}"]\ar[dr, bend right=20pt, "F"', "" name=F]&
	|[alias=CD]|\cat{C}\oplus\cat{D}\ar[d, "\copair{F,G}" description]&
	\cat{D}\ar[l, "\pair{I,\cat{D}}"']\ar[dl, bend left=20pt, "G", ""' name=G]\\&
	\cat{X}
	\ar[from=G, to=CD, phantom, near start, "\cong"]
	\ar[from=F, to=CD, phantom, near start, "\cong"]
\end{tikzcd}
\end{equation}
e.g.\ $c\otimes I\cong c$ for any $c\in\cat{C}$. The map $\copair{F,G}$ is determined (up to canonical isomorphism) by this property because every object in $\cat{C}\oplus\cat{D}$ is of the form $(c,I)\otimes(I,d)$, and similarly for morphisms. Thus we have established that $\cat{C}\oplus\cat{D}$ is the 2-categorical coproduct.

We claim it is also the 2-categorical product using the usual projections, e.g.\ $\pi_{\cat{C}}\colon\cat{C}\times\cat{D}\to\cat{C}$. These functors are easily seen to be strong monoidal. Given any symmetric monoidal category $\cat{X}$ and functors $F\colon\cat{X}\to\cat{C}$ and $G\colon\cat{X}\to\cat{D}$, we get a universal functor $\pair{F,G}\colon\cat{X}\to\cat{C}\times\cat{D}$; we need to see that if $F$ and $G$ are strong monoidal then so is $\pair{F,G}$. Indeed we have
\begin{align*}
	\pair{F,G}(x_1)\otimes\pair{F,G}(x_2)&=
	\big(F(x_1),G(x_1)\big)\otimes\big(F(x_2),G(x_2)\big)\\&=
	\big(F(x_1)\otimes F(x_2),G(x_1)\otimes G(x_2)\big)\\&\cong
	\big(F(x_1\otimes x_2),G(x_1\otimes x_2)\big)\\&=
	\pair{F,G}(x_1\otimes x_2).
\end{align*}
The product universal property diagram analogous to \cref{eqn.coproduct_smc} commutes, completing the proof that $\ssmc$ has biproducts.
\end{proof}

\begin{definition}\label{def.smf}
Let $\cat{C}$ and $\cat{D}$ be symmetric monoidal categories. Define $\smf(\cat{C},\cat{D})$ to be the symmetric monoidal category whose objects are strong monoidal functors $F\colon\cat{C}\to\cat{D}$, whose morphisms are monoidal natural transformations, and whose monoidal structure is given pointwise.
\end{definition}

The pointwise condition says that the monoidal unit in $\smf(\cat{C},\cat{D})$ is given by the constant functor at the monoidal unit of $\cat{D}$ and the monoidal product is given by $(F\otimes G)(c)\coloneqq F(c)\otimes G(c).$ This is strong monoidal because for any $c,c'\in\cat{C}$ we have natural symmetry isomorphisms
\[
	\big(F(c)\otimes G(c)\big)\otimes\big(F(c')\otimes G(c')\big)
	\cong
	\big(F(c)\otimes F(c')\big)\otimes\big(G(c')\otimes G(c')\big).
\]


\begin{proposition}
Let $\cat{C}_1,\cat{C}_2,\cat{D}_1,\cat{D}_2$ be symmetric monoidal categories. The functor
\begin{equation}\label{eqn.strict_smf_biprod}\oplus\colon\smf(\cat{C}_1,\cat{D}_1)\times\smf(\cat{C}_2,\cat{D}_2)\to\smf(\cat{C}_1\oplus\cat{C}_2,\cat{D}_1\oplus\cat{D}_2)
\end{equation}
is strict monoidal.
\end{proposition}
\begin{proof}
Since $\oplus$ is a Cartesian product (in fact biproduct), the map $\oplus$ from \cref{eqn.strict_smf_biprod} is indeed a functor. We need to check that it is strict monoidal. The monoidal unit in the domain is the pair $(I,I)$ of constant functors, and it is clearly sent to the monoidal unit $(I,I)$ in the codomain.

Suppose given $F_1, F_1'\colon\cat{C}_1\to\cat{D}_1$ and $F_2,F_2'\colon\cat{C}_2\to\cat{D}_2$. Then for any $c_1\in\cat{C}_1$ and $c_2\in\cat{C}_2$ we have equalities
\begin{align*}
	\big((F_1\otimes F_1')\oplus(F_2\otimes F_2')\big)(c_1,c_2)&=
	\big(F_1(c_1)\otimes F_1'(c_1),F_2(c_2)\otimes F_2'(c_2)\big)\\&=
	\big(F_1(c_1),F_2(c_2)\big)\otimes\big(F_1'(c_1),F_2'(c_2)\big)\\&=
	(F_1\oplus F_2)(c_1,c_2)\otimes(F_1'\oplus F_2')(c_1,c_2)\\&=
	\big((F_1\oplus F_2)\otimes(F_1'\oplus F_2')\big)(c_1,c_2)
\end{align*}
This establishes strictness, and similar calculations show that $\oplus$ is monoidal with respect to morphisms and preserves the braiding. 
\end{proof}

%======== Chapter ========%
\chapter{Supply: definition and examples}



\begin{definition}[Supply]\label{def.supply}
Let $\pp$ be a prop and $\cat{C}$ a symmetric monoidal category. A \emph{supply of $\pp$ in $\cat{C}$} consists of a strict monoidal functor $s_c\colon\pp\to\cat{C}$ for each object $c\in\cat{C}$, such that $s_c(1)=c$ and such that the following diagrams commute for every $c,d\in\cat{C}$:
\begin{equation}\label{eqn.supply_commute_tensors}
\begin{tikzcd}[column sep=55pt]
	c\tpow{m}\otimes d\tpow{m}\ar[r, "s_c(\mu)\otimes s_d(\mu)"]\ar[d, "\sigma"']&
	c\tpow{n}\otimes d\tpow{n}\ar[d, "\sigma"]\\
	(c\otimes d)\tpow{m}\ar[r, "s_{c\otimes d}(\mu)"']&
	(c\otimes d)\tpow{n}
\end{tikzcd}
\hspace{.7in}
\begin{tikzcd}
	I\ar[r, equal]\ar[d, "\sigma"']&
	I\ar[d, "\sigma"]\\
	I\tpow{m}\ar[r, "s_I(\mu)"']&
	I\tpow{n}
\end{tikzcd}
\end{equation}
We say that $f\colon c\to d$ is an \emph{$s$-homomorphism} if the following diagram commutes for all $\mu\colon m\to n$ in $\pp$:
\begin{equation}\label{eqn.nat_means_homo}
\begin{tikzcd}
	c\tpow{m}\ar[r, "s_c(\mu)"]\ar[d, "f\tpow{m}"']&
	c\tpow{n}\ar[d, "f\tpow{n}"]\\
	d\tpow{m}\ar[r, "s_d(\mu)"']&
	d\tpow{n}
\end{tikzcd}
\end{equation}
\end{definition}

\begin{example}
Let $\nn$ be the discrete prop. For any monoidal category $\cat{C}$ and object $c\in\cat{C}$, there is a unique strict monoidal functor $s_c\colon\nn\to\cat{C}$ sending $1\mapsto c$. There is nothing more to check for $s$ to be a supply, since $\nn$ is discrete, and in fact every morphism in $\cat{C}$ is an $s$-homomorphism. 

One might say that every $\cat{C}$ supplies identities, and every morphism in $\cat{C}$ is a homomorphism for identities.
\end{example}


\begin{example}\label{ex.terminal_supply}
Let $1$ denote the terminal monoidal category. For any $\pp$ there is a unique supply of $\pp$ in $1$.
\end{example}

\begin{example}\label{ex.supply_involutions}
Consider the prop $\ccat{I}$ whose morphisms are given as follows:
\[
  \ccat{I}(m,n)=
  \begin{cases}
  	\emptyset&\tn{ if }m\neq n\\
		\{\id_m, i_m\}&\tn{ if }m=n
  \end{cases}
 \]
 with $i_m\cp i_m=\id_m$ and $i_m+i_n=i_{m+n}$. If $\cat{C}$ supplies $\cat{I}$, we will say it \emph{supplies involutions}. That means that every object $c\in\cat{C}$ is equipped with an involution $i_c\colon c\to c$, compatible with tensor products in the sense that $i_{c\otimes d}=i_c\otimes i_d$.
 
For a morphism $f\colon c\to d$ to be an involution-homomorphisms just means $f$ commutes with the chosen involutions, i.e.\ $f\cp i_d=i_c\cp f$.
\end{example}

\begin{definition}[Comonoids]\label{def.prop_comonoids}
The \emph{prop for comonoids} is given by two generators
\begin{equation}\label{eqn.gen_comonoid}
\begin{tikzpicture}[WD]
	\node[bb={1}{0}, fill=gray] (eta') {};
	\draw (eta'_in1) to +(-.8,0);
	\node[bb={1}{2}, fill=gray, right=4 of eta'] (mu') {};
	\draw (mu'_in1) -- +(-.8,0);
	\draw (mu'_out1) -- +(.8,0);
	\draw (mu'_out2) -- +(.8,0);
\end{tikzpicture}
\end{equation}
and three equations:
\begin{equation}\label{eqn.rel_comonoid}
\begin{tikzpicture}
	\node (Q11) {
	\begin{tikzpicture}[WD]
		\node[bb={1}{2}, fill=gray] (a) {};
		\coordinate (a1) at ($(a_out1)+(1,0)$);
		\coordinate (a2) at ($(a_out2)+(1,0)$);
		\draw (a_out2) to[out=0, in=180] (a1);
		\draw (a_out1) to[out=0, in=180] (a2);
		\draw (a_in1) -- +(-.5,0);
	\end{tikzpicture}
	};
	\node (Q12) [right=.8 of Q11] {
	\begin{tikzpicture}[WD]
		\node[bb={1}{2}, fill=gray] (a) {};
		\draw (a_out1) -- +(.5,0);
		\draw (a_out2) -- +(.5,0);
		\draw (a_in1) -- +(-.5,0);
	\end{tikzpicture}
	};
	\node[label=above:{\tiny commutative}] at ($(Q11.east)!.5!(Q12.west)$) {$=$};
%
	\node (Q21) [right=1 of Q12] {
  \begin{tikzpicture}[WD]
  	\node[bb={1}{2}, fill=gray] (a1) {};
  	\node[bb={1}{0}, fill=gray, right=.5 of a1_out1] (a2) {};
  	\draw (a2_in1) -- (a1_out1);
  	\draw (a1_out2) -- +(2,0);
  	\draw (a1_in1) -- +(-.5,0);
	\end{tikzpicture}
	};
	\node (Q22) [right=.8 of Q21] {
	\begin{tikzpicture}[WD]
		\draw (0,0) -- (2,0);
	\end{tikzpicture}
	};	
	\node[label=above:{\tiny unital}] at ($(Q21.east)!.5!(Q22.west)$) {$=$};
%
	\node (Q31) [right=1 of Q22] {
	\begin{tikzpicture}[WD]
		\node[bb={1}{2}, fill=gray] (a1) {};
		\node[bb={1}{2}, fill=gray, minimum height=1ex, right=.5 of a1_out1] (a2) {};
		\draw (a2_in1) -- (a1_out1);
		\draw (a1_out2) -- +(2,0);
		\draw (a2_out1) -- +(.5,0);
		\draw (a2_out2) -- +(.5,0);
		\draw (a1_in1) -- +(-.5,0);
	\end{tikzpicture}
	};
	\node (Q32) [right=.8 of Q31] {
	\begin{tikzpicture}[WD]
		\node[bb={1}{2}, fill=gray] (a1) {};
		\node[bb={1}{2}, fill=gray, minimum height=1ex, right=.5 of a1_out2] (a2) {};
		\draw (a2_in1) -- (a1_out2);
		\draw (a1_out1) -- +(2,0);
		\draw (a2_out1) -- +(.5,0);
		\draw (a2_out2) -- +(.5,0);
		\draw (a1_in1) -- +(-.5,0);
	\end{tikzpicture}
	};
	\node[label=above:{\tiny associative}] at ($(Q31.east)!.5!(Q32.west)$) {$=$};
\end{tikzpicture}
\end{equation}
\end{definition}

\begin{definition}[Monoids]\label{def.prop_monoids}
The \emph{prop for monoids} is given by two generators
\begin{equation}\label{eqn.gen_monoid}
\begin{tikzpicture}[WD]
	\node[bb={0}{1}, fill=white] (eta) {};
	\draw (eta_out1) -- +(.8,0);
	\node[bb={2}{1}, fill=white, right=4 of eta] (mu) {};
	\draw (mu_out1) -- +(.8,0);
	\draw (mu_in1) -- +(-.8,0);
	\draw (mu_in2) -- +(-.8,0);
\end{tikzpicture}
\end{equation}
and three equations:
\begin{equation}\label{eqn.rel_monoid}
\begin{tikzpicture}
	\node (P11) {
	\begin{tikzpicture}[WD]
		\node[bb={2}{1}, fill=white] (a) {};
		\coordinate (a1) at ($(a_in1)-(1,0)$);
		\coordinate (a2) at ($(a_in2)-(1,0)$);
		\draw (a1) to[in=180, out=0] (a_in2);
		\draw (a2) to[in=180, out=0] (a_in1);
		\draw (a_out1) -- +(.5,0);
	\end{tikzpicture}
	};
	\node (P12) [right=.8 of P11] {
	\begin{tikzpicture}[WD]
		\node[bb={2}{1}, fill=white] (a) {};
		\draw (a_in1) -- +(-.5,0);
		\draw (a_in2) -- +(-.5,0);
		\draw (a_out1) -- +(.5,0);
	\end{tikzpicture}
	};
	\node at ($(P11.east)!.5!(P12.west)$) {$=$};
%
	\node (P21) [right=1 of P12] {
  \begin{tikzpicture}[WD]
  	\node[bb={2}{1}, fill=white] (a1) {};
  	\node[bb={0}{1}, fill=white, left=.5 of a1_in1] (a2) {};
  	\draw (a2_out1) -- (a1_in1);
  	\draw (a1_in2) -- +(-2,0);
  	\draw (a1_out1) -- +(.5,0);
	\end{tikzpicture}
	};
	\node (P22) [right=.8 of P21] {
	\begin{tikzpicture}[WD]
		\draw (0,0) -- (2,0);
	\end{tikzpicture}
	};	
	\node at ($(P21.east)!.5!(P22.west)$) {$=$};
%
	\node (P31) [right=1 of P22] {
	\begin{tikzpicture}[WD]
		\node[bb={2}{1}, fill=white] (a1) {};
		\node[bb={2}{1}, fill=white, minimum height=1ex, left=.5 of a1_in1] (a2) {};
		\draw (a2_out1) -- (a1_in1);
		\draw (a1_in2) -- +(-2,0);
		\draw (a2_in1) -- +(-.5,0);
		\draw (a2_in2) -- +(-.5,0);
		\draw (a1_out1) -- +(.5,0);
	\end{tikzpicture}
	};
	\node (P32) [right=.8 of P31] {
	\begin{tikzpicture}[WD]
		\node[bb={2}{1}, fill=white] (a1) {};
		\node[bb={2}{1}, fill=white, minimum height=1ex, left=.5 of a1_in2] (a2) {};
		\draw (a2_out1) -- (a1_in2);
		\draw (a1_in1) -- +(-2,0);
		\draw (a2_in1) -- +(-.5,0);
		\draw (a2_in2) -- +(-.5,0);
		\draw (a1_out1) -- +(.5,0);
	\end{tikzpicture}
	};
	\node at ($(P31.east)!.5!(P32.west)$) {$=$};
\end{tikzpicture}
\end{equation}
\end{definition}


\begin{example}[Supply of monoids and comonoids]
Consider the prop given by the skeleton of $\finset$, i.e. with $\Hom(m,n)\coloneqq\smset(\ord{m},\ord{n})$. Calling this prop ``the prop for monoids'' is reasonable in the sense that a supply of this prop in $\cat{C}$ gives a map $\mu_c\colon c\otimes c\to c$ and $\eta\colon I\to c$ for every object $c$, compatible with tensor product in $\cat{C}$ and satisfying the usual monoid laws. A morphism $f\colon c\to d$ is a homomorphism for monoids if $\mu_c\cp f=(f\otimes f)\cp \mu_d$ and $\eta_c\cp f=\eta_d$.

Similarly, by a supply of comonoids, we mean a supply of the prop given by the skeleton of $\finset\op$. We saw in the introduction that $\rel$ supplies comonoids.
\end{example}

\begin{example}[Cartesian categories]\label{ex.cart_grant_comonoids}
A symmetric monoidal category $\cat{C}$ has finite products iff it supplies comonoids such that every morphism in $\cat{C}$ is a comonoid homomorphism. In this case, the categorical product coincides with the monoidal product. This was shown in \cite{fox1976coalgebras}.
\end{example}

\begin{example}[Compact closed categories]
Consider the \emph{prop for self-duality} presented by two generators
\[
\begin{tikzpicture}[WD, font=\tiny, light gray nodes]
	\node[bb={2}{0}] (a) {};
	\node[bb={0}{2}, right= of a] (b) {};
	\draw (a_in1) -- +(-.5, 0);
	\draw (a_in2) -- +(-.5, 0);
	\draw (b_out1) -- +(.5, 0);
	\draw (b_out2) -- +(.5, 0);
\end{tikzpicture}
\]
and two equations
\[
\begin{tikzpicture}
	\node (P1) {
  \begin{tikzpicture}[WD, light gray nodes]
  	\node[bb={0}{2}] (a) {};
  	\node[bb={2}{0}, above right=-.75 and .5 of a] (b) {};
  	\draw (a_out1) -- (b_in2);
  	\draw (a_out2) -- +(2, 0);
  	\draw (b_in1) -- +(-2, 0);
  \end{tikzpicture}
  };
  \node (P2) [right=.7 of P1] {
  \begin{tikzpicture}[WD]
  	\draw (0,0) -- (1.5,0);
  \end{tikzpicture}
  };
  \node at ($(P1.east)!.5!(P2.west)$) {$=$};
%
	\node (P3) [right=2.5 of P2]{
  \begin{tikzpicture}[WD, light gray nodes]
  	\node[bb={0}{2}] (a) {};
  	\node[bb={2}{0}, below right=-.75 and .5 of a] (b) {};
  	\draw (a_out2) -- (b_in1);
  	\draw (a_out1) -- +(2, 0);
  	\draw (b_in2) -- +(-2, 0);
  \end{tikzpicture}
  };
  \node (P4) [right=of P3] {
  \begin{tikzpicture}[WD]
  	\draw (0,0) -- (1.5,0);
  \end{tikzpicture}
  };
  \node at ($(P3.east)!.5!(P4.west)$) {$=$};
  \node at ($(P2.east)!.5!(P3.west)$) {and};
\end{tikzpicture}
\]
A category $\cat{C}$ is called a \emph{self-dual compact closed category} if it supplies dualities.
\end{example}

\begin{example}
The prop for frobenius monoids is $\cospan$. A category supplying frobenius monoids is called a \emph{hypergraph category}; see \cite{fong2019hypergraph}.
\end{example}

%======== Chapter ========%
\chapter{Supply theory}

\Cref{thm.supply_v2} gives an alternative, slightly more compact definition of supply, but it also shows that all coherence isomorphisms---associator and unitors---are homomorphisms for the supply. Before stating it, we need the following definition, which puts all the coherence isomorphisms into a single monoidal subcategory.

%==== Section ====%
\section{Main theorems}

\begin{definition}\label{def.mob}
For any symmetric monoidal category $\cat{C}$, let $\mob{\cat{C}}\ss\cat{C}$ denote the smallest subcategory containing
\begin{enumerate}
	\item all objects of $\cat{C}$ (and identity morphisms), and
	\item all structural morphisms---unitors, associators, and braiding---from $\cat{C}$.
\end{enumerate}
Thus $\mob{\cat{C}}$ inherits a symmetric monoidal structure, and we refer to it as the \emph{symmetric monoidal category of $\cat{C}$-objects}. There is a strict monoidal functor $\inc\colon\mob{\cat{C}}\to\cat{C}$. 

Any monoidal functor $F\colon\cat{C}\to\cat{D}$ induces a monoidal functor $\mob{F}\colon\mob{\cat{C}}\to\mob{\cat{D}}$.
\end{definition}

\begin{theorem}\label{thm.supply_v2}
There is a one-to-one correspondence between supplies $s$ as in \cref{def.supply} and strict monoidal functors $s'\colon\pp\to\smf(\mob{\cat{C}},\cat{C})$ sending $1\mapsto\inc$.
\end{theorem}\begin{proof}
First note that any strict monoidal functor $s'\colon\pp\to\smf(\mob{\cat{C}},\cat{C})$ sending $1\mapsto c$ must send the object $n\in\pp$ to the $n$-fold tensor power functor, $s'(n)=(c\mapsto c\tpow{n}, f\mapsto f\tpow{n})$. Thus we write $s'(n)\coloneqq -\tpow{n}$. The laxator isomorphisms are given by symmetry isomorphisms $\sigma$:
\begin{equation}\label{eqn.symmetry_c1c2}
  \sigma\colon 
  c_1\tpow{n}\otimes c_2\tpow{n}
  \Too{\cong}
  (c_1\otimes c_2)\tpow{n}
  \qqand
  \sigma\colon I\Too{\cong}I\tpow{n}
\end{equation}
Given a morphism $\mu\colon m\to n$ in $\pp$, we obtain a monoidal natural transformation $s'(\mu)\colon s'(m)\to s'(n)$. This means that for each object $c\in\cat{C}$, we have a morphism $s'(\mu)_c\colon c\tpow{m}\to c\tpow{n}$, commuting with composition, monoidal product, and order in $\pp$: given $\nu\colon n\to p$ and $\mu'\colon m'\to n'$, the following equations hold
\begin{equation}\label{eqn.supply_composition}
  s'(\mu\cp\nu)_c=s'(\mu)_c\cp s'(\nu)_c
  \qqand
	s'(\mu+\mu')_c=s'(\mu)_c\otimes s'(\mu')_c,
\end{equation}
We also have that  $s'(\mu)$ is monoidal, i.e.\ the following diagrams commute for any $c,d\in\cat{C}$:
\begin{equation}\label{eqn.supply_commute_tensors_v2}
\begin{tikzcd}[column sep=55pt]
	c\tpow{m}\otimes d\tpow{m}\ar[r, "s'(\mu)_c\otimes s'(\mu)_d"]\ar[d, "\sigma"']&
	c\tpow{n}\otimes d\tpow{n}\ar[d, "\sigma"]\\
	(c\otimes d)\tpow{m}\ar[r, "s'(\mu)_{c\otimes d}"']&
	(c\otimes d)\tpow{n}
\end{tikzcd}
\hspace{.7in}
\begin{tikzcd}
	I\ar[r, equal]\ar[d, "\sigma"']&
	I\ar[d, "\sigma"]\\
	I\tpow{m}\ar[r, "s'(\mu)_I"']&
	I\tpow{n}
\end{tikzcd}
\end{equation}
 
It is now easy to recover a supply $s\colon\pp\to\cat{C}$ from a strict monoidal functor $s'$. On objects we put $s_c(n)\coloneqq s'(n)(c)=c\tpow{n}$ for each $n\in\nn$. On morphisms $\mu\colon m\to n$ in $\pp$, we put $s_c(\mu)\coloneqq s(\mu)_c\colon c\tpow{m}\to c\tpow{n}$. This is functorial and monoidal by \cref{eqn.supply_composition}, and the commutative diagrams \eqref{eqn.supply_commute_tensors} are just restatements of \eqref{eqn.supply_commute_tensors_v2}.

We now show the converse, that given a supply $s$ as in \cref{def.supply}, one can obtain a strict functor $s'\colon\pp\to\smf(\mob{\cat{C}},\cat{C}$. We must define $s'$ on objects so that \cref{eqn.symmetry_c1c2} holds; in particular $s'(n)\coloneqq -\tpow{n}$. Given a morphism $\mu\colon m\to n$ in $\pp$, define the $c$-component of $s'(\mu)\colon -\tpow{m}\to -\tpow{n}$ by $s'(\mu)_c\coloneqq s_c(\mu)$. Then \cref{eqn.supply_composition,eqn.supply_commute_tensors_v2} hold by the functoriality of $s_c$ and \cref{eqn.supply_commute_tensors}.
 
 The only thing that remains to is that $s'(\mu)$ is natural, i.e.\ that the diagram
 \begin{equation}\label{eqn.C_0_homos}
 \begin{tikzcd}
	c\tpow{m}\ar[d, "s_c(\mu)"']\ar[r, "f\tpow{m}"]&
	d\tpow{m}\ar[d, "s_d(\mu)"]\\
	c\tpow{n}\ar[r, "f\tpow{n}"']&
	d\tpow{n}
\end{tikzcd}
\end{equation}
commutes for any $f\colon c\to d$ in $\mob{\cat{C}}$. But the morphisms in $\mob{\cat{C}}$ are generated by the associators and unitors, so it suffices to show that these maps are homomorphisms for $s(\mu)$. For the associators we consider the following diagram:
\[
\begin{tikzcd}[row sep=30pt]
  \big((a\otimes b)\otimes c\big)\tpow{m}\ar[d, "s(\mu)_{(a\otimes b)\otimes c}"']&
  a\tpow{m}\otimes b\tpow{m}\otimes c\tpow{m}\ar[l, "\sigma"']\ar[r, "\sigma"]\ar[d, "s(\mu)_a\otimes s(\mu)_b\otimes s(\mu)_c" description]&
  \big(a\otimes (b\otimes c)\big)\tpow{m}\ar[d, "s(\mu)_{a\otimes (b\otimes c)}"]\\
  \big((a\otimes b)\otimes c\big)\tpow{n}&
  a\tpow{n}\otimes b\tpow{n}\otimes c\tpow{n}\ar[l, "\sigma"]\ar[r, "\sigma"']&
  \big(a\otimes (b\otimes c)\big)\tpow{n}
\end{tikzcd}
\]
Both the left-hand and right-hand squares commute by the left-hand diagram in \cref{eqn.supply_commute_tensors}. Replacing the leftward vertical maps by their inverses, the diagram still commutes. By Mac Lane's coherence theorem, the composite horizontal maps are the associator isomorphisms, and the fact that it commutes establishes that the associator is a homomorphism for $s(\mu)$.

The same sort of argument holds for the unitors, with one additional element: the fact that $s(\mu)_I$ is the identity. Indeed, consider the following diagram:
\[
\begin{tikzcd}[row sep=30pt]
	(a\otimes I)\tpow{m}\ar[d, "s(\mu)_{a\otimes I}"']&
	a\tpow{m}\otimes I\tpow{m}\ar[d, "s(\mu)_a\otimes s(\mu)_I" description]\ar[l, "\sigma"']&[10pt]
	a\tpow{m}\otimes I\ar[d, "s(\mu)_a\otimes I" description]\ar[r, "\rho"]\ar[l, "a\tpow{m}\otimes\sigma"']&
	a\tpow{m}\ar[d, "s(\mu)_a"]\\
	(a\otimes I)\tpow{n}&
	a\tpow{n}\otimes I\tpow{n}\ar[l, "\sigma"]&
	a\tpow{n}\otimes I\ar[l, "a\tpow{n}\otimes\sigma"]\ar[r, "\rho"']&
	a\tpow{n}
\end{tikzcd}
\]
The left-hand and middle diagrams commute by \cref{eqn.supply_commute_tensors} and the right-hand diagram commutes by the unitor axiom.
\end{proof}

\begin{theorem}\label{thm.homos_form_subcat}
Let $s$ be a supply of $\pp$ in $\cat{C}$. Then the collection of $s$-homomorphisms forms a monoidal subcategory $\mob{\cat{C}}\ss\cat{C}'\ss\cat{C}$, and the functor $s'\colon\pp\to\smf(\mob{\cat{C}},\cat{C})$ from \cref{thm.supply_v2} factors through a strict monoidal functor
\[s'\colon\pp\to\smf(\cat{C}',\cat{C}).\]
\end{theorem}
\begin{proof}
We need to show that the collection of $s$-homomorphisms $f\colon c\to d$ contains all the coherence isomorphisms in $\cat{C}$ and is closed under composition and tensor product.

We showed in \cref{thm.supply_v2} that every coherence isomorphism in $\cat{C}$ is a $s$-homomorphism; see below \cref{eqn.C_0_homos}. It is obvious that if $f\colon c\to d$ and $g\colon d\to e$ are $s$-homomorphisms then so is $f\cp g$. Finally, if $f_1\colon c_1\to d_1$ and $f_2\colon c_2\to d_2$ are $s$-homomorphisms then so is $(f_1\otimes f_2)$; this follows from \cref{eqn.supply_commute_tensors_v2}.

The final statement is just a repackaging of the statement that every morphism in $\cat{C}'$ is an $s$-homomorphism; compare \cref{eqn.nat_means_homo,eqn.C_0_homos}.
\end{proof}

\begin{theorem}
If $\cat{C}$ and $\cat{D}$ both supply $\pp$ then so does their biproduct $\cat{C}\oplus\cat{D}$.
\end{theorem}
\begin{proof}
**
\end{proof}

%==== Section ====%
\section{Further theory}


\begin{proposition}\label{prop.p_supplies_itself}
Let $\pp$ be a prop. Then there is a supply of $\pp$ in $\pp$.
\end{proposition}
\begin{proof}
The monoidal product in a prop is denoted $+$; we denote the $n$-fold monoidal product by $\cdot n$.

Given $\mu\colon m\to n$ in $\pp$, we need to define a monoidal natural transformation
\[
\begin{tikzcd}[column sep=60pt]
	\pp\ar[r, bend left, "-\cdot m" name=m]\ar[r, bend right, "-\cdot m"' name=n]&
	\pp
	\ar[from=m, to=n, Rightarrow, shorten <=4pt, shorten >=4pt, "s(\mu)"]
\end{tikzcd}
\]
We define the component $s(\mu)_k\colon k\cdot m\to k\cdot n$ for an object $k\in\pp$ by conjugating by the symmetries and applying $\mu$, on each of the $k$ pieces:
\begin{equation}\label{eqn.conjugation}
	k\cdot m\To{\sigma_{k,m}}
	m\cdot k\To{\mu\cdot k}
	n\cdot k\To{\sigma_{n,k}}
	k\cdot n.
\end{equation}
It is an easy exercise to show that the two maps $k_1\cdot m+k_2\cdot m\to (k_1+k_2)\cdot n$ from \cref{eqn.supply_commute_tensors} agree for any $k_1,k_2\in\pp$. If $\mu\leq\mu'$ then $s(\mu)_k\leq s(\mu')_k$ by whiskering \cref{eqn.conjugation}.
\end{proof}

\begin{proposition}\label{prop.change_of_supply}
Let $F\colon\pp\to\qq$ be a prop functor. For any supply $s$ of $\qq$ in $\cat{C}$, we have a supply $(F\cp s)$ of $\pp$ in $\cat{C}$.
\end{proposition}
\begin{proof}
Given a strict monoidal functor $\qq\to\smf(\mob{\cat{C}},\cat{C})$, we compose it with $F$ (which is strict) to get the required supply of $\pp$; see \cref{thm.supply_v2}.
\end{proof}

\begin{example}
  Since there is a prop functor from the prop for dualities to the prop for frobenius monoids, every hypergraph category is a self-dual compact closed category.
  \todo{Explain}
\end{example}




% Subsection %
\subsection{Preservation of supply}

\begin{proposition}\label{prop.nat_strong_isos}
Let $(F,\varphi)\colon\cat{C}\to\cat{D}$ be a strong monoidal functor. The following commutes up to natural isomorphism:
\[
\begin{tikzcd}[column sep=50pt]
	\nn\ar[r, "\otimes^-"]\ar[d, "\otimes^-"']&
	\smf(\cat{C},\cat{C})\ar[d, "{\smf(\cat{C},F)}"]\\
	\smf(\cat{D},\cat{D})\ar[r, "{\smf(F,\cat{D})}"']&
	\smf(\cat{C},\cat{D})\ar[ul, phantom, "\overset{\varphi}{\cong}"]
\end{tikzcd}
\]
\end{proposition}
\begin{proof}
Since $\nn$ is discrete, we simply need to provide an isomorphism of functors $\mob{\cat{C}}\to\cat{D}$ for each object $n\in\nn$. Along the top-right, $n$ is sent to the functor $c\mapsto F(c\tpow{n})$, and along the left-bottom, $n$ is sent to the functor $c\mapsto F(c)\tpow{n}$, and the natural isomorphisms $\varphi$ that make $F$ strong are precisely the required isomorphisms.
\end{proof}

\begin{definition}\label{def.preserve_supply}
Let $\pp$ be a prop, $\cat{C}$ and $\cat{D}$ symmetric monoidal categories, and suppose $s$ is a supply of $\pp$ in $\cat{C}$ and $t$ is a supply of $\pp$ on $\cat{D}$. We say that a strong monoidal functor $(F,\varphi)\colon\cat{C}\to\cat{D}$ \emph{preserves the supply} if there exists a natural isomorphism:
\begin{equation}\label{eqn.pres_supply}
\begin{tikzcd}[column sep=50pt]
	\pp\ar[r, "s"]\ar[d, "t"']&
	\smf(\mob{\cat{C}},\cat{C})\ar[d, "{\smf(\mob{\cat{C}},F)}"]\\
	\smf(\mob{\cat{D}},\cat{D})\ar[r, "{\smf(\mob{F},\cat{D})}"']&
	\smf(\mob{\cat{C}},\cat{D})\ar[ul, phantom, "\overset{\varphi}{\cong}"]
\end{tikzcd}
\end{equation}
\end{definition}


\begin{proposition}\label{prop.easy_pres_supply}
$F\colon\cat{C}\to\cat{D}$ preserves the supply as in \cref{def.preserve_supply} iff the following diagram commutes for each morphism $\mu\colon m\to n$ in $\pp$ and object $c\in\cat{C}$:
\begin{equation}\label{eqn.unpack_preserve_supply}
\begin{tikzcd}[column sep=large]
	F(c)\tpow{m}\ar[r, "t(\mu)_{F(c)}"]\ar[d, "\cong"']&
	F(c)\tpow{n}\ar[d, "\cong"]\\
	F(c\tpow{m})\ar[r, "F(s(\mu)_c)"']&
	F(c\tpow{n})
\end{tikzcd}
\end{equation}
\end{proposition}
\begin{proof}
For each object $m\in\pp$, there is a component isomorphism $\varphi_c\colon F(c)\tpow{m}\to F(c\tpow{m})$ natural in $c\in\mob{\cat{C}}$; this is the content of \cref{prop.nat_strong_isos}.%
\footnote{In fact, \cref{prop.nat_strong_isos} says more: that these isomorphisms $\varphi$ are natural in $\cat{C}$.\label{foot.natural}}
  For these $\varphi$ to be natural in $\pp$ means that for any morphism $\mu\colon m\to n$ in $\pp$ we have $s(\mu)\cp F=F_0\cp t(\mu)$. This is the content of \cref{eqn.unpack_preserve_supply}.
\end{proof}

\begin{proposition}
Suppose that $s\colon\pp\to\smf(\cat{C}',\cat{C})$ and $t\colon\pp\to\smf(\cat{D}',\cat{D})$ are supplies with homomorphisms in $\cat{C}'$ and $\cat{D}'$, and suppose $F\colon\cat{C}\to\cat{D}$ preserves supply as in \cref{def.preserve_supply} and furthermore restricts to a functor $F'\colon\cat{C}'\to\cat{D}'$. Then the following diagram also commutes:
\[
\begin{tikzcd}[column sep=large]
	\pp\ar[r, "s"]\ar[d, "t"']&
	\smf(\cat{C}',\cat{C})\ar[d, "{\smf(\cat{C},F)}"]\\
	\smf(\cat{D}',\cat{D})\ar[r, "{\smf(F',\cat{D})}"']&
	\smf(\cat{C}',\cat{D})\ar[ul, phantom, "\overset{\varphi}{\cong}"]
\end{tikzcd}
\]
\end{proposition}
\begin{proof}
This follows from \cref{foot.natural} and the proof of \cref{prop.easy_pres_supply}.
\end{proof}

\begin{example}
Let $s$ be a supply of $\pp$ in $\cat{C}$. Recall that there is a unique supply of $\pp$ on $1$ by \cref{ex.terminal_supply}. It follows from the second diagram in \cref{eqn.supply_commute_tensors} that the unique monoidal functor $1\to\cat{C}$ preserves the unique supply of $\pp$ on $1$.
\end{example}

\begin{example}\label{ex.preserve_involutions}
Suppose we have a supply $s$ of involutions in $\cat{C}$ and a supply $t$ of involutions in $\cat{D}$. As we saw in \cref{ex.supply_involutions} this just means that every object $x$ is equipped with an involution $i_x\colon x\cong x$. A symmetric monoidal functor $F\colon\cat{C}\to\cat{D}$ preserves the supply if $F(i_x)=i_{F(x)}$.
\end{example}

\begin{proposition}
  The strong monoidal functor of \cref{cor.supply_each_object} preserves the supply.
\end{proposition}
\begin{proof}
  \todo{Prove.}
\end{proof} 

\begin{proposition}\label{prop.strong_bo}
Suppose $F\colon\cat{C}\to\cat{D}$ is a strong monoidal functor that is bijective on objects. Then if $\cat{C}$ supplies $\cat{P}$ then so does $\cat{D}$ in such a way that $F$ preserves the supply.
\end{proposition}
\begin{proof}
Given $s$ as in the following diagram, one simply defines $t$ using the inverse of the isomorphism $\smf(\mob{F},\cat{D})$:
\[
\begin{tikzcd}[column sep=50pt]
	\pp\ar[r, "s"]\ar[d, dashed, "t"']&
	\smf(\mob{\cat{C}},\cat{C})\ar[d, "{\smf(\mob{\cat{C}},F)}"]\\
	\smf(\mob{\cat{D}},\cat{D})\ar[r, "\cong"']&
	\smf(\mob{\cat{C}},\cat{D})
\end{tikzcd}
\qedhere
\]
\end{proof}



\printbibliography
\end{document}
