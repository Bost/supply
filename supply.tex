\documentclass[11pt, oneside, article]{memoir}

\settrims{0pt}{0pt} % page and stock same size
\settypeblocksize{*}{35pc}{*} % {height}{width}{ratio}
\setlrmargins{*}{*}{1} % {spine}{edge}{ratio}
\setulmarginsandblock{1.1in}{1.4in}{*} % height of typeblock computed
\setheadfoot{\onelineskip}{2\onelineskip} % {headheight}{footskip}
\setheaderspaces{*}{1.5\onelineskip}{*} % {headdrop}{headsep}{ratio}
\checkandfixthelayout

\usepackage[centertags,sumlimits,intlimits,namelimits,reqno]{amsmath}
\usepackage{latexsym}
\usepackage{footmisc}
\usepackage{amsfonts,amsthm,amssymb}
\usepackage{mathtools}
\usepackage[cal=euler,scr=rsfso]{mathalfa}
\usepackage[boxslash]{stmaryrd}
\usepackage{newpxtext}
%\usepackage{exscale}
\usepackage{enumitem}
\usepackage{ifthen}
%\usepackage[T1]{fontenc}
%\usepackage{breakcites}
\usepackage[colorlinks,linkcolor=darkblue,citecolor=darkblue,urlcolor=darkblue,breaklinks=true]{hyperref}
\usepackage{tikz}
\usepackage[capitalize]{cleveref}
\usepackage[backend=biber, bibencoding=utf8, maxbibnames = 10, style = alphabetic]{biblatex}
\DeclareMathAlphabet{\mathpzc}{OT1}{pzc}{m}{it}
\usepackage{xstring}
\usepackage{ebproof}
\usepackage[export]{adjustbox} %vertical align includegraphics
%\usepackage{scalefnt}
%\usepackage{anyfontsize} %arbitrary font size
\usepackage{graphicx}

\usepackage[color=white, textsize=footnotesize]{todonotes}

\presetkeys%
    {todonotes}%
    {inline}{}
    
%% code from mathabx.sty and mathabx.dcl
\DeclareFontFamily{U}{mathx}{\hyphenchar\font45}
\DeclareFontShape{U}{mathx}{m}{n}{
      <5> <6> <7> <8> <9> <10>
      <10.95> <12> <14.4> <17.28> <20.74> <24.88>
      mathx10
      }{}
\DeclareSymbolFont{mathx}{U}{mathx}{m}{n}
\DeclareFontSubstitution{U}{mathx}{m}{n}
\DeclareMathAccent{\widecheck}{0}{mathx}{"71}

% amsmath %
	\allowdisplaybreaks
	
% cleveref %
  \newcommand{\creflastconjunction}{, and\nobreakspace} % serial comma
  \definecolor{darkblue}{rgb}{0,0,0.7} 


% tikz %
  \usetikzlibrary{ 
  	cd,
  	math,
  	decorations.markings,
		decorations.pathreplacing,
  	positioning,
	 	circuits.logic.US,
 		arrows.meta,
  	shapes,
		shadows,
		shadings,
  	calc,
  	fit,
  	quotes,
  	intersections,
  }

  %% rounded rectangle

\newif\ifpgfshaperectangleroundnortheast
\newif\ifpgfshaperectangleroundnorthwest
\newif\ifpgfshaperectangleroundsoutheast
\newif\ifpgfshaperectangleroundsouthwest
\pgfkeys{/pgf/.cd,
  rectangle round north east/.is if=pgfshaperectangleroundnortheast,
  rectangle round north west/.is if=pgfshaperectangleroundnorthwest,
  rectangle round south east/.is if=pgfshaperectangleroundsoutheast,
  rectangle round south west/.is if=pgfshaperectangleroundsouthwest,
  rectangle round north east, rectangle round north west,
  rectangle round south east, rectangle round south west,
}
\makeatletter
\def\pgf@sh@bg@rectangle{%
  \pgfkeysgetvalue{/pgf/outer xsep}{\outerxsep}%
  \pgfkeysgetvalue{/pgf/outer ysep}{\outerysep}%
  \pgfpathmoveto{\pgfpointadd{\southwest}{\pgfpoint{\outerxsep}{\outerysep}}}%
  {\ifpgfshaperectangleroundnorthwest\else\pgfsetcornersarced{\pgfpointorigin}\fi%
    \pgfpathlineto{\pgfpointadd{\southwest\pgf@xa=\pgf@x\northeast\pgf@x=\pgf@xa}{\pgfpoint{\outerxsep}{-\outerysep}}}}%  
  {\ifpgfshaperectangleroundnortheast\else\pgfsetcornersarced{\pgfpointorigin}\fi%  
    \pgfpathlineto{\pgfpointadd{\northeast}{\pgfpoint{-\outerxsep}{-\outerysep}}}}%
  {\ifpgfshaperectangleroundsoutheast\else\pgfsetcornersarced{\pgfpointorigin}\fi%  
    \pgfpathlineto{\pgfpointadd{\southwest\pgf@ya=\pgf@y\northeast\pgf@y=\pgf@ya}{\pgfpoint{-\outerxsep}{\outerysep}}}}%
  {\ifpgfshaperectangleroundsouthwest\else\pgfsetcornersarced{\pgfpointorigin}\fi%  
    \pgfpathclose}}


%% WD classes

\tikzset{
  	WD/.style={%everything after equals replaces "oriented WD" in key.
  	  label/.style={
      	font=\everymath\expandafter{\the\everymath\scriptstyle},
        inner sep=0pt,
        node distance=2pt and -2pt},
      semithick,
      node distance=\bbx and \bby,
      decoration={markings, mark=at position \stringdecpos with \stringdec},
    	bb port length=0,
  	  bb port sep=.5,
	 	  bbx = .4cm,
		  bb min width=.4cm,
	    bby = 2ex,
	    bb penetrate=0,
	    bb rounded corners=2pt,
	    dot size=3pt,
      pack size = 16pt,
    	penetration = 0pt,
      link size = 2pt,
      pack color = blue,
      surround sep=2pt,
      ar/.style={postaction={decorate}},
  		execute at begin picture={\tikzset{
  			x=\bbx, y=\bby, 
				circuit logic US, tiny circuit symbols
				}
			}
    },
    bbx/.store in=\bbx,
    bby/.store in=\bby,
    bb port sep/.store in=\bbportsep,
    bb port length/.store in=\bbportlen,
    bb penetrate/.store in=\bbpenetrate,
    bb min width/.store in=\bbminwidth,
    bb rounded corners/.store in=\bbcorners,
    bb/.code 2 args={%When you see this key, run the code below:
    	\pgfmathsetlengthmacro{\bbheight}{\bbportsep * (max(#1,#2)+1) * \bby}
      \pgfkeysalso{draw,minimum height=\bbheight,minimum
       width=\bbminwidth,outer sep=0pt,
         rounded corners=\bbcorners,thick,
         prefix after command={\pgfextra{\let\fixname\tikzlastnode}},
         append after command={\pgfextra{\draw
            \ifnum #1=0{} \else foreach \i in {1,...,#1} {
            	($(\fixname.north west)!{(2*\i-1)/(2*#1)}!(\fixname.south west)$) +(-\bbportlen,0) coordinate (\fixname_in\i) -- +(\bbpenetrate,0) coordinate (\fixname_in\i')}\fi 
  					%Define the endpoints of tickmarks
            \ifnum #2=0{} \else foreach \i in {1,...,#2} {
            	($(\fixname.north east)!{(2*\i-1)/(2*#2)}!(\fixname.south east)$) +(-
\bbpenetrate,0) coordinate (\fixname_out\i') -- +(\bbportlen,0) coordinate (\fixname_out\i)}\fi;
           }}}
		},
	dot size/.store in=\dotsize,
	dot/.style={
		circle, draw, thick, inner sep=0, fill=black, minimum width=\dotsize
	},
	pack size/.store in=\psize,
	penetration/.store in=\penetration,
  spacing/.store in=\spacing,
  link size/.store in=\lsize,
  pack color/.store in=\pcolor,
 	pack inside color/.store in=\picolor,
  pack inside color=blue!20,
 	pack outside color/.store in=\pocolor,
  pack outside color=blue!50!black,
 	surround sep/.store in=\ssep,
 	link/.style={
  	circle, 
  	draw=black, 
  	fill=black,
  	inner sep=0pt, 
 		minimum size=\lsize
 	},
  pack/.style={
 		circle, 
 		draw = \pocolor, 
  	fill = \picolor,
  	minimum size = \psize
  },
  func/.style={
  	pack,
		rectangle,
		rounded corners=.5*\psize,
		inner ysep=.125*\psize,
		minimum width=1.125*\psize,
		inner xsep=.25*\psize,
  },
  funcr/.style={
    func,
    rectangle round north west=false, 
		rectangle round south west=false,
  },
  funcl/.style={
    func,
		rectangle round north east=false, 
		rectangle round south east=false,
  },
  funcu/.style={
    func,
		rectangle round south east=false, 
		rectangle round south west=false,
  },
  funcd/.style={
    func,
		rectangle round north east=false, 
		rectangle round north west=false,
  },
  outer pack/.style={
 		ellipse, 
 		draw,
  	inner sep=\ssep,
  	color=gray,
 	},
  intermediate pack/.style={
 		ellipse,
 		dashed, 
  	draw,
  	inner sep=\ssep,
 		color=\pocolor,
 	},
 }
 
\tikzset{light gray nodes/.style={every node/.style={fill=gray!40}}}


\tikzset{
	oriented WD/.style={%everything after equals replaces "oriented WD" in key.
		every to/.style={out=0,in=180,draw},
    label/.style={
    	font=\everymath\expandafter{\the\everymath\scriptstyle},
      inner sep=0pt,
      node distance=2pt and -2pt},
    semithick,
    node distance=1 and 1,
    decoration={markings, mark=at position \stringdecpos with \stringdec},
    ar/.style={postaction={decorate}},
    execute at begin picture={\tikzset{
    	x=\bbx, y=\bby,
      every fit/.style={inner xsep=\bbx, inner ysep=\bby}}}
    },
    string decoration/.store in=\stringdec,
    string decoration={\arrow{stealth};},
    string decoration pos/.store in=\stringdecpos,
    string decoration pos=.7,
    bbx/.store in=\bbx,
    bbx = 1.5cm,
    bby/.store in=\bby,
    bby = 1.5ex,
    bb port sep/.store in=\bbportsep,
    bb port sep=1.5,
    % bb wire sep/.store in=\bbwiresep,
    % bb wire sep=1.75ex,
    bb port length/.store in=\bbportlen,
    bb port length=4pt,
    bb penetrate/.store in=\bbpenetrate,
    bb penetrate=0,
    bb min width/.store in=\bbminwidth,
    bb min width=1cm,
    bb rounded corners/.store in=\bbcorners,
    bb rounded corners=2pt,
    bb spider/.style={
    	bb port sep=1, bb port length=10pt, bbx=.4cm, bb min width=.4cm, bby=.8ex},
    bb small/.style={
    	bb port sep=1, bb port length=2.5pt, bbx=.4cm, bb min width=.4cm, bby=.7ex},
		bb medium/.style={
			bb port sep=1, bb port length=2.5pt, bbx=.4cm, bb min width=.4cm, bby=.9ex},
    bb/.code 2 args={%When you see this key, run the code below:
    	\pgfmathsetlengthmacro{\bbheight}{\bbportsep * (max(#1,#2)+1) * \bby}
      \pgfkeysalso{draw,minimum height=\bbheight,minimum
       width=\bbminwidth,outer sep=0pt,
         rounded corners=\bbcorners,thick,
         prefix after command={\pgfextra{\let\fixname\tikzlastnode}},
         append after command={\pgfextra{\draw
            \ifnum #1=0{} \else foreach \i in {1,...,#1} {
            	($(\fixname.north west)!{(2*\i-1)/(2*#1)}!(\fixname.south west)$) +(-\bbportlen,0) coordinate (\fixname_in\i) -- +(\bbpenetrate,0) coordinate (\fixname_in\i')}\fi 
  					%Define the endpoints of tickmarks
            \ifnum #2=0{} \else foreach \i in {1,...,#2} {
            	($(\fixname.north east)!{(2*\i-1)/(2*#2)}!(\fixname.south east)$) +(-
\bbpenetrate,0) coordinate (\fixname_out\i') -- +(\bbportlen,0) coordinate (\fixname_out\i)}\fi;
           }}}
		},
		bb name/.style={
    	append after command={
				\pgfextra{\node[anchor=north] at (\fixname.north) {#1};}
			}
		},
  }


\tikzset{
	unoriented WD/.style={
  	every to/.style={draw},
  	shorten <=-\penetration, shorten >=-\penetration,
  	label distance=-2pt,
  	thick,
  	node distance=\spacing,
  	execute at begin picture={\tikzset{
  		x=\spacing, y=\spacing, circuit logic US, tiny circuit symbols}
		}
  },
  pack size/.store in=\psize,
  pack size = 12pt,
	penetration/.store in=\penetration,
	penetration = 0pt,
  spacing/.store in=\spacing,
  spacing = 8pt,
  link size/.store in=\lsize,
  link size = 2pt,
  pack color/.store in=\pcolor,
  pack color = blue,
 	pack inside color/.store in=\picolor,
  pack inside color=blue!20,
 	pack outside color/.store in=\pocolor,
  pack outside color=blue!50!black,
 	surround sep/.store in=\ssep,
  surround sep=8pt,
 	link/.style={
  	circle, 
  	draw=black, 
  	fill=black,
  	inner sep=0pt, 
 		minimum size=\lsize
 	},
  pack/.style={
 		circle, 
 		draw = \pocolor, 
  	fill = \picolor,
  	minimum size = \psize
  },
  func/.style={
  	pack,
		rectangle,
		rounded corners=.5*\psize,
		inner ysep=.125*\psize,
		minimum width=1.125*\psize,
		inner xsep=.25*\psize,
  },
  funcr/.style={
    func,
    rectangle round north west=false, 
		rectangle round south west=false,
  },
  funcl/.style={
    func,
		rectangle round north east=false, 
		rectangle round south east=false,
  },
  funcu/.style={
    func,
		rectangle round south east=false, 
		rectangle round south west=false,
  },
  funcd/.style={
    func,
		rectangle round north east=false, 
		rectangle round north west=false,
  },
  outer pack/.style={
 		ellipse, 
 		draw,
  	inner sep=\ssep,
  	color=gray,
 	},
  intermediate pack/.style={
 		ellipse,
 		dashed, 
  	draw,
  	inner sep=\ssep,
 		color=\pocolor,
 	},
}
  
\tikzset{
	spider diagram/.style={
		every to/.style={out=0, in=180, draw, thick},
		thick,
		dot size = 5pt,
		execute at begin picture={\tikzset{
    	x=\leglen, y=\leglen/3}}
	},
	dot size/.store in=\dotsize,
	dot fill/.store in=\dotfill,
	dot fill = black,
	leg length/.store in=\leglen,
	leg length = 15pt,
	baby/.style={dot size = 2pt, leg length = 6pt},
	young/.style={dot size = 3pt, leg length = 10pt},
	adolescent/.style={dot size = 4pt, leg length = 12pt},
	special spider/.code n args={4}{
		\pgfkeysalso{circle, draw, thick, inner sep=0, fill=\dotfill, minimum width=\dotsize,
  		prefix after command={\pgfextra{\let\fixname\tikzlastnode}},
  		append after command={\pgfextra{
  			\ifnum #1=0{} \else {\foreach \i in {1,...,#1} {
					\tikzmath{\anglei={-90*(#1+1-2*\i)/#1};}
  				\draw [thick]
						(\fixname) .. controls 
						($(\fixname.center)-(\anglei:#3/3)$) and ($(\fixname.center)-(\anglei:#3*2/3)$) .. 
						({$(\fixname.center)-(\anglei:#3*2/3)$}-|{$(\fixname.center)-(#3,0)$}) coordinate (\fixname_in\i);
  			}}\fi
  			\ifnum #2=0{} \else {\foreach \i in {1,...,#2} {
					\tikzmath{\anglei={90*(#2+1-2*\i)/#2};}
  				\draw [thick]
						(\fixname.center) .. controls 
						($(\fixname.center)+(\anglei:#4/3)$) and ($(\fixname.center)+(\anglei:#4*2/3)$) .. 
						({$(\fixname.center)+(\anglei:#4*2/3)$}-|{$(\fixname.center)+(#4,0)$}) coordinate (\fixname_out\i);
  			}}\fi
  		}}
		}
	},
	spider/.code 2 args={
		\pgfkeysalso{special spider={#1}{#2}{\leglen}{\leglen}}
	}
}

\tikzset{
	inner WD/.style={
		every to/.style={out=0, in=180, draw, thick},
		unoriented WD, 
		surround sep=2pt, 
		font=\tiny, 
		anchor=center
	}
}

\tikzset{
  function/.style={->, thin, shorten <=4pt, shorten >=4pt}
}

\tikzset{
  tick/.style={
  	postaction={
    	decorate,
      decoration={
      	markings, mark=at position 0.5 with {
					\draw[-] (0,.4ex) -- (0,-.4ex);
				}
			}
		}
	}
} 

\newcommand{\tickar}{\begin{tikzcd}[baseline=-0.5ex,cramped,sep=small,ampersand 
replacement=\&]{}\ar[r,tick]\&{}\end{tikzcd}}

\tikzcdset{
	twocell/.style={Rightarrow, shorten <=5pt, shorten >=5pt}
}
 
\newcommand{\simpletheta}{
\begin{tikzpicture}[unoriented WD, surround sep=2pt, pack size=6pt, font=\tiny, baseline=(theta.base)]
	\node[pack] (theta) {$\theta$};
	\draw (theta.west) -- +(-2pt, 0);
	\draw (theta.east) -- +(2pt, 0);
\end{tikzpicture}
}

\newcommand{\dectheta}{\begin{tikzpicture}[unoriented WD, surround sep=2pt, font=\tiny, pack size=6pt, baseline=(theta.base)]
	\node[pack] (theta) {$\theta$};
	\draw (theta.west) to[pos=1] node[left=-3pt] {$\Gamma_1$} +(-2pt, 0);
	\draw (theta.east) to[pos=1] node[right=-3pt] {$\Gamma_2$} +(2pt, 0);
\end{tikzpicture}}


	
% biblatex %
  \addbibresource{Library20190610.bib} 

% enumitem %
  \setlist{noitemsep, nolistsep}
	\setlist[description]{leftmargin=0em, itemindent=2em}
		
%-------------------------------------------------------------------------
\theoremstyle{plain}
\newtheorem{theorem}{Theorem}[chapter] %change [] to chapter if we want to change global numbering
\newtheorem{proposition}[theorem]{Proposition}
\newtheorem{corollary}[theorem]{Corollary}
\newtheorem{lemma}[theorem]{Lemma}
\newtheorem{conjecture}[theorem]{Conjecture}

\theoremstyle{definition}
\newtheorem{definition}[theorem]{Definition}
\newtheorem{construction}[theorem]{Construction}
\newtheorem{notation}[theorem]{Notation}
\newtheorem{axiom}{Axiom}
\newtheorem*{axiom*}{Axiom}

\theoremstyle{remark}
\newtheorem{example}[theorem]{Example}
\newtheorem{remark}[theorem]{Remark}
\newtheorem{warning}[theorem]{Warning}
\newtheorem{question}[theorem]{Question}

\crefalias{chapter}{section}

%------------------Begin author macros-----------------------

\newcommand{\Set}[1]{\mathrm{#1}}%a named set
\newcommand{\ord}[1]{\underline{#1}}%a natural number, considered as a finite set
\newcommand{\const}[1]{\mathtt{#1}}%a constant, named element of a set, sort of thing
\newcommand{\cat}[1]{\mathcal{#1}}%a generic category
\newcommand{\ccat}[1]{\mathbb{#1}}%a generic po-category
\newcommand{\Cat}[1]{{\mathsf{#1}}}%a named category
\newcommand{\CCat}[1]{\mathbb{\StrLeft{#1}{1}}\Cat{\StrGobbleLeft{#1}{1}}}%a named po-category; does not seem to work in section headers...
\newcommand{\funn}[1]{\mathrm{#1}}%a function
\newcommand{\funr}[1]{\mathcal{#1}}%a generic functor
\newcommand{\Funr}[1]{\mathsf{#1}}%a named functor
\newcommand{\ffunr}[1]{\mathbf{#1}}%a generic 2-functor


\DeclareMathOperator{\ob}{\Set{Ob}}
\DeclareMathOperator{\id}{id}
\DeclareMathOperator{\dom}{dom}
\DeclareMathOperator{\cod}{cod}
\DeclareMathOperator{\Hom}{Hom}
\DeclareMathOperator*{\colim}{colim}
\DeclareMathOperator{\coker}{coker}
\DeclareMathOperator{\im}{im}
\DeclareMathOperator{\inc}{inc}
\DeclarePairedDelimiter{\pair}{\langle}{\rangle}
\DeclarePairedDelimiter{\unary}{{\langle\,}}{{\,\rangle}}
\DeclarePairedDelimiter{\copair}{[}{]}
\DeclarePairedDelimiter{\classify}{{\raisebox{1pt}{$\ulcorner$}}}{{\raisebox{1pt}{$\urcorner$}}}
\DeclarePairedDelimiter{\church}{\xxbracket}{\rrbracket}

\newcommand{\tn}[1]{\textnormal{#1}}
\newcommand{\op}{^{\tn{op}}}
\newcommand{\inv}{^{-1}}
\newcommand{\tpow}[1]{^{\otimes #1}}

\newcommand{\finset}{\Cat{FinSet}}
\newcommand{\smset}{\Cat{Set}}
\newcommand{\poset}{\Cat{Poset}}
\newcommand{\ssmf}{\CCat{SMF}}
\newcommand{\ssmc}{\CCat{SMC}}

\renewcommand{\aa}{\mathbb{A}} %Not sure what old \aa did
\renewcommand{\ll}{\mathbb{L}} %Old \ll is <<
\newcommand{\cc}{\mathbb{C}}
\newcommand{\dd}{\mathbb{D}}
\newcommand{\nn}{\mathbb{N}}
\newcommand{\pp}{\mathbb{P}}
\newcommand{\qq}{\mathbb{Q}}
\newcommand{\xx}{\mathbb{X}} 
\newcommand{\zz}{\mathbb{Z}}

\newcommand{\mob}[1]{#1_0}



\newcommand{\cp}{\mathbin{\fatsemi}}
\newcommand{\cocolon}{:\!}
\newcommand{\iso}{\cong}
\newcommand{\too}{\longrightarrow}
\newcommand{\tto}{\rightrightarrows}
\newcommand{\To}[1]{\xrightarrow{#1}}
\newcommand{\Too}[1]{\To{\;\;#1\;\;}}
\newcommand{\from}{\leftarrow}
\newcommand{\From}[1]{\xleftarrow{#1}}
\newcommand{\Fromm}[1]{\xleftarrow{\;\;#1\;\;}}
\newcommand{\tofrom}{\leftrightarrows}
\newcommand{\surj}{\twoheadrightarrow}
\newcommand{\inj}{\rightarrowtail}
\newcommand{\Surj}[1]{\overset{#1}{\twoheadrightarrow}}
\newcommand{\Inj}[1]{\overset{#1}{\rightarrowtail}}
\newcommand{\frsurj}{\twoheadleftarrow}
\newcommand{\frinj}{\leftarrowtail}
\newcommand{\ul}[1]{\underline{#1}}
\renewcommand{\ss}{\subseteq}


\newcommand{\qqand}{\qquad\text{and}\qquad}
\newcommand{\qand}{\quad\text{and}\quad}

\newcommand{\hide}[2][]{#1}

\newcommand{\adjphantom}[3][-.6pt]{\ar[#2, phantom, "#3" yshift=#1]}
\newcommand{\adj}[5][30pt]{%[size] Cat L, Left, Right, Cat R.
\begin{tikzcd}[ampersand replacement=\&, column sep=#1]
  #2\ar[r, shift left=5pt, "{#3}"]\adjphantom{r}{\Rightarrow}\&
  #5\ar[l, shift left=5pt, "{#4}"]
\end{tikzcd}
}

\newcommand{\adjr}[5][30pt]{%[size] Cat R, Right, Left, Cat L.
\begin{tikzcd}[ampersand replacement=\&, column sep=#1]
  #2\ar[r, shift left=5pt, "{#3}"]\adjphantom{r}{\Leftarrow}\&
  #5\ar[l, shift left=5pt, "{#4}"]
\end{tikzcd}
}

\newcommand{\pb}[1][very near start]{\ar[dr, phantom, #1, "\lrcorner"]}
\newcommand{\po}[1][very near start]{\ar[ul, phantom, #1, "\ulcorner"]}

\setlength\tabcolsep{3pt}
\linespread{1.10}

%================ Document ================%
\begin{document}   

\title{Supplying bells and whistles in monoidal categories}
\author{Brendan Fong$^*$ \and David I.\ Spivak\thanks{Spivak and Fong acknowledge support from AFOSR grants FA9550-17-1-0058 and FA9550-19-1-0113.}}
  
\maketitle

\tableofcontents*

%======== Chapter ========%
\chapter{Introduction}

\section{Notation}\label{sec.notation}
For a natural number $n\in\nn$ we denote the corresponding ordinal by $\ord{n}=\{1,\ldots,n\}\in\smset$. Given a finite set $S$, we write $|S|\in\nn$ for its cardinality.

We denote composition of $f\colon a\to b$ and $g\colon b\to c$ by $(f\cp g)\colon a\to c$, i.e.\ we use diagrammatic order. When using application order, we write $g\circ f$. When $c$ is an object we denote the identity morphism on it either by $c$ or by $\id_c$.

Given morphisms $f\colon A\to C$ and $g\colon B\to C$ in a category with coproducts, we denote the copairing of $f$ and $g$ by $\copair{f,g}\colon(A\sqcup B)\to C$. Similarly if the category has products, we denote the pairing of $f\colon A\to B$ and $g\colon A\to C$ by $\pair{f,g}\colon A\to B\times C$.

%======== Chapter ========%
\chapter{Background}

%==== Section ====%
\section{Monoidal po-categories}

We use the prefix \emph{po-} as an abbreviation for \emph{locally posetal}.

\begin{definition}[Po-category]
By a \emph{po-category} we mean a locally-posetal 2-category. A \emph{po-functor} is simply a 2-functor between po-categories.
\end{definition}
A po-category $\cc$ can also be thought of as a category enriched in $(\poset,\times,\ord{1})$. More explicitly, $\cc$ has, for every pair of objects $c,c'\in\cc$, a poset $\cc(c,c')$, where we denote the order relation by $\leq$. Composition $\cp\colon\cc(c,c')\times \cc(c',c'')\to\cc(c,c'')$ in a po-category is required to be monotonic: $f\leq g$ and $f'\leq g'$ imply $(f\cp f')\leq(g\cp g')$. From this perspective, every po-category has an underlying category, and a po-functor is a functor between the underlying categories that is required to preserve the order on morphisms. A \emph{natural transformation} between po-functors is just a natural transformation between the underlying functors.

\begin{definition}[Symmetric monoidal po-category]
A \emph{symmetric monoidal structure} on a po-category $\cc$ consists of a pair $(\otimes,I)$ where
\begin{enumerate}
	\item $\otimes\colon\cc\times\cc\to\cc$ is a po-functor,
	\item $I\in\cc$ is an object,
\end{enumerate}
together with an associator, left and right unitor, and braiding isomorphisms satisfying the usual axioms of symmetric monoidal categories.

A \emph{lax monoidal structure} on po-functor $F\colon\cc\to\cc'$ consists of a morphism  $\varphi\colon I'\to F(I)$ and a natural maps $\varphi_{c_1,c_2}\colon F(c_1)\otimes' F(c_2)\to F(c_1\otimes c_2)$, that appropriately commute with the associators, unitors, and braiding. We say that $F$ is \emph{strong monoidal} if $\varphi$ and each $\varphi_{c_1,c_2}$ is an isomorphism, and we say $F$ is strict if these are all identities.
\end{definition}

\begin{definition}[$\ssmf(\cc,\dd)$]\label{def.smf}
Let $\cc$ and $\dd$ be symmetric monoidal po-categories. Define a symmetric monoidal po-category $\ssmf(\cc,\dd)$ whose objects are strong monoidal functors, whose 1-morphisms are monoidal natural transformations, whose 2-morphisms are modifications, and whose monoidal structure is given pointwise.
\end{definition}
We unpack the definition a bit. An object of $\ssmf(\cc,\dd)$ is a strong monoidal functor $F\colon\cc\to\dd$, and a morphism $\alpha\colon F\to G$ is a monoidal natural transformation. We say that $\alpha\leq\alpha'$ if for every object $c\in\cc$ one has an inequality
\[
\begin{tikzcd}
	F(c)\ar[r, bend left, "\alpha_c"]\ar[r, bend right, "\alpha'_c"']\ar[r, phantom, "\Downarrow"]&
	G(c)
\end{tikzcd}
\]
in $\dd(F(c),G(c)$. Finally, the pointwise condition says that the monoidal unit in $\ssmf(\cc,\dd)$ is given by the constant functor at the monoidal unit of $\dd$ and the monoidal product is given by $(F\otimes G)(c)\coloneqq F(c)\otimes G(c).$ This is strong monoidal because for any $c,c'\in\cc$ we have natural symmetry isomorphisms
\[
	\big(F(c)\otimes G(c)\big)\otimes\big(F(c')\otimes G(c')\big)
	\cong
	\big(F(c)\otimes F(c')\big)\otimes\big(G(c')\otimes G(c')\big).
\]

\begin{theorem}
The 2-category $\ssmc$ of symmetric monoidal categories, strong monoidal functors, and monoidal natural transformations has strong 2-biproducts. The same is true for symmetric monoidal po-categories and strong monoidal po-functors.
\end{theorem}
\begin{proof}
The terminal category is symmetric monoidal, and it is a zero object
Let $\cat{C}$ and $\cat{D}$ be symmetric monoidal categories. We will show that their product $\cat{C}\times\cat{D}$ as categories is in fact a symmetric monoidal category, and that it is their biproduct; we denote it $\cat{C}\oplus\cat{D}\coloneqq\cat{C}\times\cat{D}$.

As a symmetric monoidal structure on $\cat{C}\oplus\cat{D}$, take $(I,I)$ to be the monoidal unit and $(c_1,d_1)\otimes(c_2,d_2)\coloneqq(c_1\otimes c_2,d_1\otimes d_2)$ to be the monoidal product. The associators, unitors, and braiding are given pointwise.

The functor $\pair{\cat{C},I}\colon\cat{C}\to\cat{C}\oplus\cat{D}$ sending $c\mapsto (c,I)$ is clearly strong monoidal. We claim that it and $\pair{I,\cat{D}}$ form the coprojections under which $\cat{C}\oplus\cat{D}$ is the coproduct. Indeed, given strong monoidal functors $F\colon\cat{C}\to\cat{X}$ and $G\colon\cat{D}\to\cat{X}$, define $\copair{F,G}\colon\cat{C}\oplus\cat{D}\to\cat{X}$ by $\copair{F,G}(c,d)\coloneqq F(c)\otimes G(d)$, and similarly for morphisms. This is strong monoidal using the isomorphisms
\begin{align*}
	\copair{F,G}(c_1,d_1)\otimes\copair{F,G}(c_2,d_2)&=
	F(c_1)\otimes G(d_1)\otimes F(c_2)\otimes G(d_2)\\&\cong
	F(c_1)\otimes F(c_2)\otimes G(d_1)\otimes G(d_2)\\&\cong
	\copair{F,G}\big((c_1,d_1)\otimes(c_2,d_2)\big).
\end{align*}
where the first isomorphism is the braiding in $\cat{C}$ and the second isomorphism uses the (strong) laxators from $F$ and $G$. It is straightforward to check that this satisfies the necessary axioms to be a laxator%
\hide[.]{
, e.g.\ the commutativity of the following diagram
\[
\begin{tikzcd}
	Fc_1\otimes Gd_1\otimes Fc_2\otimes Gd_2\otimes Fc_3\otimes Gd_3\ar[r, "\sigma"]\ar[d, "\sigma"']&
	Fc_1\otimes Gd_1\otimes Fc_2\otimes Fc_3\otimes Gd_2\otimes Gd_3\ar[d]\\
	Fc_1\otimes Fc_2\otimes Gc_1\otimes Gc_2\otimes Fc_3\otimes Gd_3\ar[d]&
	Fc_1\otimes Gd_1\otimes F(c_2\otimes c_3)\otimes G(d_2\otimes d_3)\ar[d, "\sigma"]\\
	F(c_1\otimes c_2)\otimes G(d_1\otimes d_2)\otimes Fc_3\otimes Gd_3\ar[d, "\sigma"']&
	Fc_1\otimes F(c_2\otimes c_3)\otimes Gd_1\otimes G(d_2\otimes d_3)\ar[d]\\
	F(c_1\otimes c_2)\otimes Fc_3\otimes G(d_1\otimes d_2)\otimes Gd_3\ar[r]&
	F(c_1\otimes c_2\otimes c_3)\otimes G(d_1\otimes d_2\otimes d_3)	
\end{tikzcd}
\]
}
It is also easy to check that the unitors provide natural isomorphisms
\begin{equation}\label{eqn.coproduct_smc}
\begin{tikzcd}[sep=large]
	\cat{C}\ar[r, "\pair{\cat{C},I}"]\ar[dr, bend right=20pt, "F"', "" name=F]&
	|[alias=CD]|\cat{C}\oplus\cat{D}\ar[d, "\copair{F,G}" description]&
	\cat{D}\ar[l, "\pair{I,\cat{D}}"']\ar[dl, bend left=20pt, "G", ""' name=G]\\&
	\cat{X}
	\ar[from=G, to=CD, phantom, near start, "\cong"]
	\ar[from=F, to=CD, phantom, near start, "\cong"]
\end{tikzcd}
\end{equation}
e.g.\ $c\otimes I\cong c$ for any $c\in\cat{C}$. The map $\copair{F,G}$ is determined (up to canonical isomorphism) by this property because every object in $\cat{C}\oplus\cat{D}$ is of the form $(c,I)\otimes(I,d)$, and similarly for morphisms. Thus we have established that $\cat{C}\oplus\cat{D}$ is the 2-categorical coproduct.

We claim it is also the 2-categorical product using the usual projections, e.g.\ $\pi_{\cat{C}}\colon\cat{C}\times\cat{D}\to\cat{C}$. These functors are easily seen to be strong monoidal. Given any symmetric monoidal category $\cat{X}$ and functors $F\colon\cat{X}\to\cat{C}$ and $G\colon\cat{X}\to\cat{D}$, we get a universal functor $\pair{F,G}\colon\cat{X}\to\cat{C}\times\cat{D}$; we need to see that if $F$ and $G$ are strong monoidal then so is $\pair{F,G}$. Indeed we have
\begin{align*}
	\pair{F,G}(x_1)\otimes\pair{F,G}(x_2)&=
	\big(F(x_1),G(x_1)\big)\otimes\big(F(x_2),G(x_2)\big)\\&=
	\big(F(x_1)\otimes F(x_2),G(x_1)\otimes G(x_2)\big)\\&\cong
	\big(F(x_1\otimes x_2),G(x_1\otimes x_2)\big)\\&=
	\pair{F,G}(x_1\otimes x_2).
\end{align*}
The product universal property diagram analogous to \cref{eqn.coproduct_smc} commutes, completing the proof that $\ssmc$ has biproducts.

Finally, if $\cat{C}$ and $\cat{D}$ are replaced by symmetric monoidal po-categories $\ccat{C}$ and $\ccat{D}$, the proof runs analogously.
\end{proof}

\begin{proposition}
Let $\ccat{C}_1,\ccat{C}_2,\ccat{D}_1,\ccat{D}_2$ be symmetric monoidal po-categories. The po-functor
\begin{equation}\label{eqn.strict_smf_biprod}\oplus\colon\ssmf(\ccat{C}_1,\ccat{D}_1)\times\ssmf(\ccat{C}_2,\ccat{D}_2)\to\ssmf(\ccat{C}_1\oplus\ccat{C}_2,\ccat{D}_1\oplus\ccat{D}_2)
\end{equation}
is strict monoidal.
\end{proposition}
\begin{proof}
Since $\oplus$ is a Cartesian product (in fact biproduct), the map $\oplus$ from \cref{eqn.strict_smf_biprod} is indeed a po-functor. We need to check that it is strict monoidal. The monoidal unit in the domain is the pair $(I,I)$ of constant functors, and it is clearly sent to the monoidal unit $(I,I)$ in the codomain.

Suppose given $F_1, F_1'\colon\ccat{C}_1\to\ccat{D}_1$ and $F_2,F_2'\colon\ccat{C}_2\to\ccat{D}_2$. Then for any $c_1\in\ccat{C}_1$ and $c_2\in\ccat{C}_2$ we have equalities
\begin{align*}
	\big((F_1\otimes F_1')\oplus(F_2\otimes F_2')\big)(c_1,c_2)&=
	\big(F_1(c_1)\otimes F_1'(c_1),F_2(c_2)\otimes F_2'(c_2)\big)\\&=
	\big(F_1(c_1),F_2(c_2)\big)\otimes\big(F_1'(c_1),F_2'(c_2)\big)\\&=
	(F_1\oplus F_2)(c_1,c_2)\otimes(F_1'\oplus F_2')(c_1,c_2)\\&=
	\big((F_1\oplus F_2)\otimes(F_1'\oplus F_2')\big)(c_1,c_2)
\end{align*}
This establishes strictness, and similar calculations show that $\oplus$ is monoidal with respect to morphisms and preserves the braiding. 
\end{proof}

\begin{definition}\label{def.mob}
For any symmetric monoidal po-category $\cc$, let $\mob{\cc}\ss\cc$ denote the smallest sub po-category containing
\begin{enumerate}
	\item all objects of $\cc$ (and identity morphisms and 2-morphisms), and
	\item all structural 1-morphisms---unitors, associators, and braiding---from $\cc$.
\end{enumerate}
Thus $\mob{\cc}$ is symmetric monoidal, and locally discrete. We refer to it as the \emph{symmetric monoidal category of $\cc$-objects}. There is a strict monoidal functor $\inc\colon\mob{\cc}\to\cc$. 

Any monoidal po-functor $F\colon\cc\to\dd$ induces a monoidal po-functor $\mob{F}\colon\mob{\cc}\to\mob{\dd}$.
\end{definition}

Suppose $\cat{C}$ is a symmetric monoidal category, $m\in\nn$ is a natural number, and $c\colon\ord{m}\to\cat{C}$ is a family of objects in $\cat{C}$. We denote
\[
  \bigotimes c\coloneqq\bigotimes_{i\in\ord{m}}c(i)\coloneqq
  \big((c(1)\otimes c(2))\cdots\big)\otimes c(m)
\]
with the convention that when $m=0$ and $!\colon\ord{0}\to \cat{C}$ the unique function, we put $\bigotimes != I$. If $c(i)=c(j)$ for all $i,j\in\ord{m}$, we denote this by $c\tpow{m}\coloneqq\bigotimes_{i\in\ord{m}}c$. We take this to be the canonical parenthesation, so $c\otimes d\otimes e$ denotes $(c\otimes d)\otimes e$.

If $m,n\in\nn$ are natural numbers, and $c\colon \ord{m}\times \ord{n}\to\cc$ is a family of objects in $\cc$, we also have a natural isomorphism
\begin{equation}\label{eqn.symmetry}
\sigma\colon
\bigotimes_{i\in\ord{m}}\bigotimes_{j\in\ord{n}}c(i,j)\Too{\cong}
\bigotimes_{j\in\ord{n}}\bigotimes_{i\in\ord{m}}c(i,j).
\end{equation}
We refer to $\sigma$ as the \emph{symmetry} isomorphism, though note that it involves associators and unitors too, not just the symmetric braiding. We will be interested in two particular cases of the symmetry isomorphism \cref{eqn.symmetry}, namely for $m=2$ and $m=0$ and any $n\in\nn$:
\[\sigma\colon c_1\tpow{n}\otimes c_2\tpow{n}\Too{\cong}(c_1\otimes c_2)\tpow{n}
\qqand
\sigma\colon I\Too{\cong} I\tpow{n}
\]

%==== Section ====%
\section{Po-props and typed po-props}

% Subsection %
\subsection{Definitions and examples}

\begin{definition}[Props]\label{def.props}
A \emph{po-prop} is a strict monoidal po-category $\pp$ whose monoid of objects is equal to $(\nn,0,+)$. 
\end{definition}

\begin{example}\label{ex.nat_prop}
The initial one-typed po-prop is the discrete symmetric monoidal category $(\nn,0,+)$.
\end{example}

\begin{example}\label{ex.involutions}
Consider the prop $\ccat{I}$ whose morphisms are given as follows:
\[
  \ccat{I}(m,n)=
  \begin{cases}
  	\emptyset&\tn{ if }m\neq n\\
		\{\id_m, i_m\}&\tn{ if }m=n
  \end{cases}
 \]
 with $i_m\cp i_m=\id_m$ and $i_m+i_n=i_{m+n}$. 
\end{example}

For any $m,n\in\nn$ and prop $\pp$, \cref{eqn.symmetry} provides an isomorphism
\[\sigma\colon mn\To{\cong}nm,\]
where $mn=m+m+\cdots+m$, $n$-many times, and $nm=n+n+\cdots+n$, $m$-many times.


%======== Chapter ========%
\chapter{Supply: equipping each object with algebraic structure}

%==== Section ====%
\section{Supply and homomorphic supply}

It is often the case that each object of a monoidal category $\cc$ is equipped with a certain structure, compatible with the monoidal product. For example, perhaps every object is equipped with a monoid structure. Some authors say that $\cc$ ``has'' the structure, e.g.\ ``has monoids''. In this section we define this notion more formally, give some examples, and prove some properties of it.

Recall from \cref{def.mob} the strict monoidal functor $\mob{\cc}\to\cc$ including the monoidal category of $\cc$ objects into $\cc$. It gives rise to a strict monoidal functor $\ssmf(\cc,\cc)\to\ssmf(\mob{\cc},\cc)$, where $\ssmf(-,-)$ is the po-category of strong monoidal functors and monoidal natural transformations from \cref{def.smf}. We will be interested in the strict monoidal functor $^{\otimes-}\colon\nn\to\ssmf(\mob{\cc},\cc)$.

The results of this section will rely on Mac Lane's coherence theorem for symmetric monoidal categories \cite[Theorem XI.1]{maclane:1998a}, which says the following. For any two ways to arrange parentheses and monoidal units into a word with $n$ placeholders for objects in $\cat{C}$, and for each permutation of $n$ letters, there is an associated natural isomorphism, which Mac Lane calls the \emph{canonical isomorphism}, between the resulting functors $\cat{C}^n\to\cat{C}$, and composites and tensor products of canonical isomorphisms are again canonical. For example, everything we called a symmetry isomorphisms $\sigma$ in \cref{eqn.symmetry} is one of these canonical isomorphisms. Finally, nothing changes if we replace $\cat{C}$ by a symmetric monoidal po-category $\cc$, because by definition all diagrams of 2-cells commute in $\cc$.

\begin{proposition}\label{prop.homomorphically_supply_objects}
Let $(\cc,\otimes,I)$ be a symmetric monoidal po-category and let $\cc_0\ss\cc$ be its monoidal category of $\cc$-objects. There is a unique strict monoidal functor $\tpow{-}\colon\nn\to\ssmf(\cc_0,\cc)$ sending $1\mapsto\id$. Moreover, it factors through a strict functor $\nn\to\ssmf(\cc,\cc)$.
\end{proposition}
\begin{proof}
By \cref{def.smf}, any strict monoidal functor $\nn\to\ssmf(\cc_0,\cc)$ sending $1\mapsto\id$ must send objects $n\in\nn$ to the $n$-fold tensor power functor, $c\mapsto c\tpow{n}$  and $f\mapsto f\tpow{n}$ for $f\colon c\to d$ in $\cc_0$. The laxator isomorphisms are given by symmetry isomorphisms $\sigma$:
\begin{equation}\label{eqn.symmetry_c1c2}
  \sigma\colon 
  c_1\tpow{n}\otimes c_2\tpow{n}
  \Too{\cong}
  (c_1\otimes c_2)\tpow{n}
  \qqand
  \sigma\colon I\Too{\cong}I\tpow{n}
\end{equation}
This assignment is functorial in $\nn$ because $\nn$ is discrete. Finally, it factors through a strict functor $\nn\to\ssmf(\cc,\cc)$ because the mapping $f\mapsto f\tpow{n}$ is functorial for any $f$ in $\cc$.
\end{proof}

\begin{remark}
Every symmetric monoidal category (or prop) is a locally-discrete symmetric monoidal po-category (or po-prop), i.e.\ we can see it as having 2-morphisms, all of which are identities. Thus we may apply \cref{def.supply} in the non-po case. On a first pass, a reader may drop the ``po'' throughout the following definition.
\end{remark}

\begin{definition}\label{def.supply}
Let $\pp$ be a po-prop and $\cc$ a symmetric monoidal po-category. A \emph{supply of $\pp$ in $\cc$} is a strict monoidal functor
\[
  s\colon\pp\to\ssmf(\mob{\cc},\cc).
\]
If $s$ factors as $\pp\to\ssmf(\cc',\cc)\ss\ssmf(\cc_0,\cc)$ for a monoidal subcategory $\cc_0\ss\cc'\ss\cc$ then we say that \emph{morphisms in $\cc'$ are homomorphisms for $s$}.
\end{definition}

Let's unpack the notion of supply $s\colon\pp\to\ssmf(\cc_0,\cc)$ from \cref{def.supply}; we deal with homomorphisms afterwards. It follows from \cref{prop.homomorphically_supply_objects,ex.nat_prop} that for every object $n\in\pp$, the strong monoidal functor $s(n)$ sends $c\mapsto c\tpow{n}$ and gives rise to the isomorphisms in \cref{eqn.symmetry_c1c2}. It remains to understand the action of $s$ on the 1-morphisms and 2-morphisms in $\pp$.

Given a 1-morphism $\mu\colon m\to n$ in $\pp$, we obtain a monoidal natural transformation $s(\mu)\colon s(m)\to s(n)$, the \emph{$\mu$-maps supplied by $s$}. This means that for each object $c\in\cc$, we have a morphism $s(\mu)_c\colon c\tpow{m}\to c\tpow{n}$, commuting with composition, monoidal product, and order in $\pp$: given $\nu\colon n\to p$ and $\mu'\colon m'\to n'$, the following equations hold
\begin{equation}\label{eqn.supply_composition}
  s(\mu\cp\nu)_c=s(\mu)_c\cp s(\nu)_c
  \qqand
	s(\mu+\mu')_c=s(\mu)_c\otimes s(\mu')_c,
\end{equation}
and given a 2-morphism $\mu_1\leq\mu_2$ in $\pp$ there must be a 2-morphism in $\cc$:
\begin{equation}\label{eqn.supply_2morphism}
\begin{tikzcd}
	c\tpow{m}\ar[r, bend left, "s(\mu_1)_c"]\ar[r, bend right, "s(\mu_2)_c"']\ar[r, phantom, "\Downarrow"]&
	c\tpow{n}.
\end{tikzcd}
\end{equation}

A priori, we must also check that the naturality condition holds for $s(\mu)$, which says that each morphism in $\cc_0$, i.e.\ each associator $\alpha\colon (a\otimes b)\otimes c\to a\otimes b\otimes c$ and unitor $\rho\colon a\otimes I\to a$ and $\lambda\colon I\otimes a\to a$, is a homomorphism for $s(\mu)$. But these conditions in fact follow from \cref{eqn.supply_commute_tensors}, as we will see in \cref{thm.coherence_are_homos}. Thus the only remaining requirement on $s(\mu)$ is that it be monoidal, i.e.\ that the diagrams
\begin{equation}\label{eqn.supply_commute_tensors}
\begin{tikzcd}[column sep=55pt]
	c\tpow{m}\otimes d\tpow{m}\ar[r, "s(\mu)_c\otimes s(\mu)_d"]\ar[d, "\sigma"']&
	c\tpow{n}\otimes d\tpow{n}\ar[d, "\sigma"]\\
	(c\otimes d)\tpow{m}\ar[r, "s(\mu)_{c\otimes d}"']&
	(c\otimes d)\tpow{n}
\end{tikzcd}
\hspace{.7in}
\begin{tikzcd}
	I\ar[r, equal]\ar[d, "\sigma"']&
	I\ar[d, "\sigma"]\\
	I\tpow{m}\ar[r, "s(\mu)_I"']&
	I\tpow{n}
\end{tikzcd}
\end{equation}
 commute for every $c,d\in\cc$. The first diagram says that the supplied morphisms commute with the monoidal product in $\cc$ and the second diagram says that the morphisms supplied on $I$ are all identities, up to the canonical morphisms $I\to I\tpow{m}$. 

If the supply $s$ factors through some $\ssmf(\cc',\cc)$, we say that morphisms $f\colon c\to d$ in $\cc'$ are homomorphisms for $s$ because the added requirement is that the following square commutes for each $\mu\colon m\to n$ in $\pp$:
\begin{equation}\label{eqn.nat_means_homo}
\begin{tikzcd}
	c\tpow{m}\ar[r, "s(\mu)_c"]\ar[d, "f\tpow{m}"']&
	c\tpow{n}\ar[d, "f\tpow{n}"]\\
	d\tpow{m}\ar[r, "s(\mu)_d"']&
	d\tpow{n}
\end{tikzcd}
\end{equation}

\begin{remark}\label{rem.notation_supply}
The notation for supplies should be as lightweight as possible. Given a supply $s$ of $\pp$, we often denote the component $s(\mu)_c$ simply by $\mu_c$, for a morphism $\mu\in\pp$ and an object $c\in\cc$.
\end{remark}

\begin{theorem}\label{thm.coherence_are_homos}
If $s$ satisfies \cref{eqn.supply_composition,eqn.supply_2morphism,eqn.supply_commute_tensors} then it is a supply.
\end{theorem}
We begin by clarifying the theorem statement. Define a \emph{pre-natural transformation} between functors $\ccat{C}\to\dd$ to be the \emph{data} of a natural transformation---a component map in $\dd$ for each object in $\ccat{C}$---\emph{without the naturality condition}. Say that a pre-natural transformation is \emph{strong monoidal} if it satisfies the usual additional strong monoidality axiom, and let $\ssmf'(\cc,\dd)$ denote the category of functors $\cc\to\dd$ and strong monoidal pre-natural transformations. Clearly, there is an inclusion $\ssmf(\cc,\dd)\to\ssmf'(\cc,\dd)$. Define a \emph{pre-supply} to be a strict monoidal functor $s'\colon\pp\to\ssmf'(\mob{\cc},\cc)$, i.e.\ $s'$ is an assignment $(\mu,c)\mapsto s'(\mu)_c\colon c\tpow{m}\to c\tpow{n}$ satisfying \cref{eqn.supply_composition,eqn.supply_2morphism,eqn.supply_commute_tensors}. The statement of \cref{thm.coherence_are_homos} says that every pre-supply is a supply, i.e.\ that every pre-supply $s'$ factors through the inclusion $\ssmf(\mob{\cc},\cc)\to\ssmf'(\mob{\cc},\cc)$ and thus induces a supply $s\colon\pp\to\ssmf(\mob{\cc},\cc)$. We are now ready to give the proof. 
\begin{proof}
Choose any $\mu\colon m\to n$ in $\pp$ and assume \cref{eqn.supply_composition,eqn.supply_2morphism,eqn.supply_commute_tensors}. The only thing that remains to check in order for $s$ to constitute a supply is that the associators and unitors are homomorphisms for $s(\mu)$, because these morphisms generate $\mob{\cc}$ by \cref{def.mob}. For the associators we consider the following diagram:
\[
\begin{tikzcd}[row sep=30pt]
  \big((a\otimes b)\otimes c\big)\tpow{m}\ar[d, "s(\mu)_{(a\otimes b)\otimes c}"']&
  a\tpow{m}\otimes b\tpow{m}\otimes c\tpow{m}\ar[l, "\sigma"']\ar[r, "\sigma"]\ar[d, "s(\mu)_a\otimes s(\mu)_b\otimes s(\mu)_c" description]&
  \big(a\otimes (b\otimes c)\big)\tpow{m}\ar[d, "s(\mu)_{a\otimes (b\otimes c)}"]\\
  \big((a\otimes b)\otimes c\big)\tpow{n}&
  a\tpow{n}\otimes b\tpow{n}\otimes c\tpow{n}\ar[l, "\sigma"]\ar[r, "\sigma"']&
  \big(a\otimes (b\otimes c)\big)\tpow{n}
\end{tikzcd}
\]
Both the left-hand and right-hand squares commute by the left-hand diagram in \cref{eqn.supply_commute_tensors}. Replacing the leftward vertical maps by their inverses, the diagram still commutes. By Mac Lane's coherence theorem, the composite horizontal maps are the associator isomorphisms, and the fact that it commutes establishes that the associator is a homomorphism for $s(\mu)$.

The same sort of argument holds for the unitors, with one additional element: the fact that $s(\mu)_I$ is the identity. Indeed, consider the following diagram:
\[
\begin{tikzcd}[row sep=30pt]
	(a\otimes I)\tpow{m}\ar[d, "s(\mu)_{a\otimes I}"']&
	a\tpow{m}\otimes I\tpow{m}\ar[d, "s(\mu)_a\otimes s(\mu)_I" description]\ar[l, "\sigma"']&[10pt]
	a\tpow{m}\otimes I\ar[d, "s(\mu)_a\otimes I" description]\ar[r, "\rho"]\ar[l, "a\tpow{m}\otimes\sigma"']&
	a\tpow{m}\ar[d, "s(\mu)_a"]\\
	(a\otimes I)\tpow{n}&
	a\tpow{n}\otimes I\tpow{n}\ar[l, "\sigma"]&
	a\tpow{n}\otimes I\ar[l, "a\tpow{n}\otimes\sigma"]\ar[r, "\rho"']&
	a\tpow{n}
\end{tikzcd}
\]
The left-hand and middle diagrams commute by \cref{eqn.supply_commute_tensors} and the right-hand diagram commutes by the unitor axiom.
\end{proof}

% Subsection %
\subsection{Examples of supply}



\begin{example}
Taking $\pp=\nn$, we see that there is a unique supply of $\nn$ in any symmetric monoidal po-category $\cc$. One might say that every $\cc$ supplies identities, and every morphism in $\cc$ is a homomorphism for identities.
\end{example}


\begin{example}\label{ex.terminal_supply}
Let $1$ denote the terminal monoidal category. For any $\pp$ there is a unique supply of $\pp$ in $1$.
\end{example}

\begin{example}\label{ex.supply_involutions}
Recall the prop $\ccat{I}$ from \cref{ex.involutions}. If $\cc$ supplies $\ccat{I}$, we will say it \emph{supplies involutions}. That means that every object $c\in\cc$ is equipped with an involution $i_c\colon c\to c$, compatible with tensor products in the sense that $i_{c\otimes d}=i_c\otimes i_d$.
 
If morphisms $f\colon c\to d$ in some $\cc'\ss\cc$ are homomorphisms for the supply, it means they commute with the chosen involutions, i.e.\ $f\cp i_d=i_c\cp f$.
\end{example}

\begin{example}[Supply of monoids and comonoids]
Consider the prop given by the skeleton of $\finset$, i.e. with $\Hom(m,n)\coloneqq\smset(\ord{m},\ord{n})$. Calling this prop ``the prop for monoids'' is reasonable in the sense that a supply of this prop in $\cc$ gives a map $\mu_c\colon c\otimes c\to c$ and $\eta\colon I\to c$ for every object $c$, compatible with tensor product in $\cc$ and satisfying the usual monoid laws. A morphism $f\colon c\to d$ is a homomorphism for monoids if $\mu_c\cp f=(f\otimes f)\cp \mu_d$ and $\eta_c\cp f=\eta_d$.

Similarly, by a supply of comonoids, we mean a supply of the prop given by the skeleton of $\finset\op$.
\end{example}

\begin{example}[Cartesian categories]\label{ex.cart_grant_comonoids}
A symmetric monoidal category $\cat{C}$ has finite products iff it supplies comonoids such that every morphism in $\cat{C}$ is a comonoid homomorphism. In this case, the categorical product coincides with the monoidal product. This was shown in \cite{fox1976coalgebras}.
\end{example}

\begin{example}[Compact closed categories]
  A category $\cat{C}$ is called a \emph{self-dual compact closed category} if it supplies dualities (\cref{def.prop_duality}).
\end{example}

\begin{remark} \label{rem.name}
  In any self-dual compact closed category $(\cat{C},\otimes,I)$, we obtain a bijection
 \[\classify{-}\colon\cat{C}(X,Y)\To{\cong}\cat{C}(I,X\otimes Y).\]
 Given $f \colon X \to Y$, we refer to $\classify{f}\colon I \to X \otimes Y$ as the \emph{name} of $f$. In pictures the construction $\classify{-}$ and its inverse are shown as follows:
\[
\begin{tikzpicture}
 	\node (p1) {
	\begin{tikzpicture}[WD, light gray nodes]
		\node[bb={1}{1}] (f) {$f$};
		\draw (f_in1) -- +(-.5, 0);
		\draw (f_out1) -- +(.5, 0);
	\end{tikzpicture}
	};
	\node (p2) [right=1 of p1] {
	\begin{tikzpicture}[WD, light gray nodes, font=\tiny]
		\node[bb port sep=.75, bb={0}{2}] (cup) {};
		\node[bb={1}{1}, right=.5 of cup_out2] (f) {$f$};
		\draw (f_in1) -- (cup_out2);
		\draw (f_out1) -- +(.5, 0);
		\draw (cup_out1) -- +(2, 0);
	\end{tikzpicture}
	};
	\node at ($(p1.east)!.5!(p2.west)$) {$\mapsto$};
%
	\node (p3) [right=2 of p2] {
	\begin{tikzpicture}[WD, light gray nodes]
		\node[bb={0}{2}] (g) {$g$};
		\draw (g_out1) -- +(.5, 0);
		\draw (g_out2) -- +(.5, 0);
	\end{tikzpicture}
	};
	\node (p4) [right=1 of p3] {
	\begin{tikzpicture}[WD, light gray nodes, font=\tiny]
		\node[bb={0}{2}] (g) {$g$};
		\node[bb={2}{0}, above right=-.75 and .5 of g] (cap) {};
		\draw (g_out1) -- (cap_in2);
		\draw (g_out2) -- +(2, 0);
		\draw (cap_in1) -- +(-2, 0);
	\end{tikzpicture}
	};
	\node at ($(p3.east)!.5!(p4.west)$) {$\mapsto$};
\end{tikzpicture}
\]
\end{remark}


% Subsection %
\subsection{First facts about supply}

We record a few more facts about supplies.

\begin{proposition}\label{prop.p_supplies_itself}
Let $\pp$ be a po-prop. Then there is a supply of $\pp$ in $\pp$.
\end{proposition}
\begin{proof}
The monoidal product in a prop is denoted $+$; we denote the $n$-fold monoidal product by $\cdot n$.

Given $\mu\colon m\to n$ in $\pp$, we need to define a monoidal natural transformation
\[
\begin{tikzcd}[column sep=60pt]
	\pp\ar[r, bend left, "-\cdot m" name=m]\ar[r, bend right, "-\cdot m"' name=n]&
	\pp
	\ar[from=m, to=n, Rightarrow, shorten <=4pt, shorten >=4pt, "s(\mu)"]
\end{tikzcd}
\]
We define the component $s(\mu)_k\colon k\cdot m\to k\cdot n$ for an object $k\in\pp$ by conjugating by the symmetries and applying $\mu$, on each of the $k$ pieces:
\begin{equation}\label{eqn.conjugation}
	k\cdot m\To{\sigma_{k,m}}
	m\cdot k\To{\mu\cdot k}
	n\cdot k\To{\sigma_{n,k}}
	k\cdot n.
\end{equation}
It is an easy exercise to show that the two maps $k_1\cdot m+k_2\cdot m\to (k_1+k_2)\cdot n$ from \cref{eqn.supply_commute_tensors} agree for any $k_1,k_2\in\pp$. If $\mu\leq\mu'$ then $s(\mu)_k\leq s(\mu')_k$ by whiskering \cref{eqn.conjugation}.
\end{proof}

\begin{remark}
One might imagine that all morphisms in $\pp$ should be homomorphisms for the supply of $\pp$ in $\pp$ but this does not generally hold; roughly speaking, it requires too much commutativity. Explicitly, there is no reason to expect the following diagram (from \cref{eqn.nat_means_homo}) to commute for $\mu\colon m\to n$ and $\delta\colon k_1\to k_2$:
\[
\begin{tikzcd}
  k_1\cdot m\ar[r, "\sigma"]\ar[d, "\delta\cdot m"']&
  m\cdot k_1\ar[r, "\mu\cdot k_1"]\ar[dr, phantom, "?"]&
  n\cdot k_1\ar[r, "\sigma"]&
  k_1\cdot n\ar[d, "\delta\cdot n"]\\
  k_2\cdot m\ar[r, "\sigma"']&
  m\cdot k_2\ar[r, "\mu\cdot k_1"']&
  n\cdot k_2\ar[r, "\sigma"']&
  k_2\cdot n
\end{tikzcd}
\]
\end{remark}


The following is straightforward.
\begin{proposition}\label{prop.change_of_supply}
Let $F\colon\ccat{P}\to\ccat{Q}$ be a prop functor. For any supply $s$ of $\ccat{Q}$ in $\cc$, we have a supply $(F\cp s)$ of $\pp$ in $\cc$.
\end{proposition}

\begin{example}
  Since there is a prop functor from the prop for dualities to the prop for frobenius monoids, every hypergraph category is a self-dual compact closed category.
\end{example}

\begin{proposition}\label{prop.supply_each_object}
A supply $s$ of $\pp$ in $\cc$ induces a strong monoidal po-functor $s_c\colon\pp\to\cc$ with $s_c(1)=c$ for every object $c\in\cc$. 

Conversely, an $\ob(\cc)$-indexed collection of strong monoidal po-functors $s_c\colon\pp\to\cc$ constitutes a supply if it satisfies \cref{eqn.supply_commute_tensors}.
\end{proposition}
\begin{proof}
Suppose given a supply $s$; we construct $s_c$ for $c\in\cc$ as follows. On objects we put $s_c(n)\coloneqq c\tpow{n}$ for each $n\in\nn$. On morphisms we put $s_c(\mu)\coloneqq s(\mu)_c$, for each $\mu\in\pp(m,n)$. This is functorial and monoidal by \cref{eqn.supply_composition}.

Suppose given a collection of such functors $s_c$, one for each $c\in\cc$, satisfying \cref{eqn.supply_commute_tensors}. By \cref{thm.coherence_are_homos}, these constitute a supply $s\colon\pp\to\ssmf(\cc_0,\cc)$ by defining $s(m)_c\coloneqq s_c(m)$ and $s(\mu)_c\coloneqq s_c(\mu)$.
\end{proof}

\begin{corollary} \label{cor.supply_each_object}
A supply $s$ of $\pp$ in $\cc$ induces a strong monoidal functor $\bigsqcup_{c\in\ob(\cc)}\pp\too\cc$.
\end{corollary}
\begin{proof}
This follows from \cref{prop.prop_coprod,prop.supply_each_object}.
\end{proof}

% Subsection %
\subsection{Preservation of supply}

\begin{proposition}\label{prop.nat_strong_isos}
Let $(F,\varphi)\colon\cc\to\dd$ be a strong monoidal functor. The following commutes up to natural isomorphism:
\[
\begin{tikzcd}[column sep=50pt]
	\nn\ar[r, "\otimes^-"]\ar[d, "\otimes^-"']&
	\ssmf(\cc,\cc)\ar[d, "{\ssmf(\cc,F)}"]\\
	\ssmf(\dd,\dd)\ar[r, "{\ssmf(F,\dd)}"']&
	\ssmf(\cc,\dd)\ar[ul, phantom, "\overset{\varphi}{\cong}"]
\end{tikzcd}
\]
\end{proposition}
\begin{proof}
Since $\nn$ is discrete, we simply need to provide an isomorphism of functors $\cc_0\to\dd$ for each object $n\in\nn$. Along the top-right, $n$ is sent to the functor $c\mapsto F(c\tpow{n})$, and along the left-bottom, $n$ is sent to the functor $c\mapsto F(c)\tpow{n}$, and the natural isomorphisms $\varphi$ that make $F$ strong are precisely the required isomorphisms.
\end{proof}

\begin{definition}\label{def.preserve_supply}
Let $\pp$ be a prop, $\cc$ and $\dd$ symmetric monoidal categories, and suppose $s$ is a supply of $\pp$ in $\cc$ and $t$ is a supply of $\pp$ on $\dd$. We say that a strong monoidal functor $(F,\varphi)\colon\cc\to\dd$ \emph{preserves the supply} if there exists a natural isomorphism:
\begin{equation}\label{eqn.pres_supply}
\begin{tikzcd}[column sep=50pt]
	\pp\ar[r, "s"]\ar[d, "t"']&
	\ssmf(\mob{\cc},\cc)\ar[d, "{\ssmf(\mob{\cc},F)}"]\\
	\ssmf(\mob{\dd},\dd)\ar[r, "{\ssmf(\mob{F},\dd)}"']&
	\ssmf(\mob{\cc},\dd)\ar[ul, phantom, "\overset{\varphi}{\cong}"]
\end{tikzcd}
\end{equation}
\end{definition}


\begin{proposition}\label{prop.easy_pres_supply}
$F\colon\ccat{C}\to\dd$ preserves the supply as in \cref{def.preserve_supply} iff the following diagram commutes for each morphism $\mu\colon m\to n$ in $\pp$ and object $c\in\ccat{C}$:
\begin{equation}\label{eqn.unpack_preserve_supply}
\begin{tikzcd}[column sep=large]
	F(c)\tpow{m}\ar[r, "t(\mu)_{F(c)}"]\ar[d, "\cong"']&
	F(c)\tpow{n}\ar[d, "\cong"]\\
	F(c\tpow{m})\ar[r, "F(s(\mu)_c)"']&
	F(c\tpow{n})
\end{tikzcd}
\end{equation}
\end{proposition}
\begin{proof}
For each object $m\in\pp$, there is a component isomorphism $\varphi_c\colon F(c)\tpow{m}\to F(c\tpow{m})$ natural in $c\in\cc_0$; this is the content of \cref{prop.nat_strong_isos}.%
\footnote{In fact, \cref{prop.nat_strong_isos} says more: that these isomorphisms $\varphi$ are natural in $\cc$.\label{foot.natural}}
  For these $\varphi$ to be natural in $\pp$ means that for any morphism $\mu\colon m\to n$ in $\pp$ we have $s(\mu)\cp F=F_0\cp t(\mu)$. This is the content of \cref{eqn.unpack_preserve_supply}.
\end{proof}

\begin{proposition}
Suppose that $s\colon\pp\to\ssmf(\cc',\cc)$ and $t\colon\pp\to\ssmf(\dd',\dd)$ are supplies with homomorphisms in $\cc'$ and $\dd'$, and suppose $F\colon\cc\to\dd$ preserves supply as in \cref{def.preserve_supply} and furthermore restricts to a functor $F'\colon\cc'\to\dd'$. Then the following diagram also commutes:
\[
\begin{tikzcd}[column sep=large]
	\pp\ar[r, "s"]\ar[d, "t"']&
	\ssmf(\cc',\cc)\ar[d, "{\ssmf(\cc,F)}"]\\
	\ssmf(\dd',\dd)\ar[r, "{\ssmf(F',\dd)}"']&
	\ssmf(\cc',\dd)\ar[ul, phantom, "\overset{\varphi}{\cong}"]
\end{tikzcd}
\]
\end{proposition}
\begin{proof}
This follows from \cref{foot.natural} and the proof of \cref{prop.easy_pres_supply}.
\end{proof}

\begin{example}
Let $s$ be a supply of $\pp$ in $\cc$. Recall that there is a unique supply of $\pp$ on $1$ by \cref{ex.terminal_supply}. It follows from the second diagram in \cref{eqn.supply_commute_tensors} that the unique monoidal functor $1\to\cc$ preserves the unique supply of $\pp$ on $1$.
\end{example}

\begin{example}\label{ex.preserve_involutions}
Suppose we have a supply $s$ of involutions in $\cc$ and a supply $t$ of involutions in $\dd$. As we saw in \cref{ex.supply_involutions} this just means that every object $x$ is equipped with an involution $i_x\colon x\cong x$. A symmetric monoidal functor $F\colon\cc\to\dd$ preserves the supply if $F(i_x)=i_{F(x)}$.
\end{example}

\begin{proposition}
  The strong monoidal functor of \cref{cor.supply_each_object} preserves the supply.
\end{proposition}
\begin{proof}
  \todo{Prove.}
\end{proof} 

\begin{proposition}\label{prop.strong_bo}
Suppose $F\colon\cc\to\dd$ is a strong monoidal functor that is bijective on objects. Then if $\cc$ supplies $\ccat{P}$ then so does $\dd$ in such a way that $F$ preserves the supply.
\end{proposition}
\begin{proof}
Given $s$ as in the following diagram, one simply defines $t$ using the inverse of the isomorphism $\ssmf(\mob{F},\dd)$:
\[
\begin{tikzcd}[column sep=50pt]
	\pp\ar[r, "s"]\ar[d, dashed, "t"']&
	\ssmf(\mob{\cc},\cc)\ar[d, "{\ssmf(\mob{\cc},F)}"]\\
	\ssmf(\mob{\dd},\dd)\ar[r, "\cong"']&
	\ssmf(\mob{\cc},\dd)
\end{tikzcd}
\qedhere
\]
\end{proof}



\printbibliography
\end{document}
